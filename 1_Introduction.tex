\chapter{Introduction}


本章从AI发展的角度阐述了深度学习是什么,从哪里来,可以用来解决什么问题。

自人工智能发展之初的符号逻辑时代起,人们就借助于规则去解决一些常规方法所不能解决的问题。虽然符号逻辑可以解决一些简单的问题,但是对于一些复杂的情况,如何用符号逻辑去表示这些情况,却比解决问题本身更加困难。之后发展到机器学习时代,人们直接从原始数据中提取出特征,用特征训练模型来解决问题。这就相当于从符号逻辑时代向前更进了一步。但是如何从原始数据中选择合适的特征是一个充满经验性和技巧性的环节。从这一点出发,就发展出了表示学习(repersentation learning)。表示学习通过学习的手段从原始数据中提取出特征,而非传统的手工选取以及手工加工组合特征的方式。深度学习就是表示学习的一种。

深度学习中的深度是指模型的深度较大,具体可以从两个方面进行阐述:
\begin{itemize}
\item 模型的算法流程执行环节较多
\item 描述各个概念关联的图的深度较大
\end{itemize}

但是并没有一个确切的指标表示达到什么样的标准才叫做深度模型。所以深度模型可以泛指包含大量通过学习而得的函数或者通过学习而得的概念的模型,这里的大量是与传统的机器学习中的函数或者概念相对而言的。

基于以上,机器学习就是:
\begin{itemize}
\item 向AI方向更近一步的方式
\item 一种机器学习的方法
\end{itemize}

\section{Who shoule read this book?}

本书的结构为

\begin{itemize}
\item \textbf{Part I} 一些基本概念
\item \textbf{Part II} 一些基本的深度学习算法,以及相应的训练方法
\item \textbf{Part III} 一些前瞻性的想法,并且这些想法极有可能对未来深度学习的发展起到推动性的作用。
\end{itemize}

由于时间有限,本书的重点将放在Part II,其他的部分快速带过。

\section{Historical Treands in Deep Learning}

本节主要介绍了深度学习的发展历史。
\begin{enumerate}
\item 深度学习有着长久的历史;
\item 深度学习近年来有广泛的应用场景是因为可供使用的学习数据增多了;
\item 深度学习模型的复杂度也在不断增加;
\item 深度学习方法解决的问题复杂度也在不断增加,同时准确率也在不断增加。
\end{enumerate}