\chapter{Deep Feedforward Networks}

\textit{深度前向神经网络}(Deep Feedforward Network)是深度神经网络的经典模型,它通过学习参数$\theta$来估计映射函数$y=f(\mathbf x)$.在前向神经网络中,信息向前流动没有反馈,有反馈的网络被称为\textit{recurrent neural network}(RNN).从工程角度看,前向神经网络是许多工业用途的神经网络的基础,因而它十分重要.前向深度神经网络通过有向无环图直观表示如何将不同的函数进行组合,进而形成网络.

在发展之初,神经网络是一个神经科学模型,但如今它在数学和工程领域内得以发展,也就脱离了原始的神经科学的范畴.

如果想要理解神经网络,最好从线性模型及其不足之处开始,逐步过渡到神经网络领域中.通常如果要将线性模型扩展到非线性领域,常用映射$\phi$为原始数据提供一种新的表示方法.常见的映射有三种形式:
\begin{enumerate}
\item 使用通用的映射$\phi$(如RBF)
\item 人工设计映射$\phi$
\item 使用机器学习的方法学习出映射$\phi$
\end{enumerate}

使用前向神经网络以外的其他模型学习特征的基本准则是深度学习的一项重要内容.

\section{Example: Learning XOR}

本节通过举例使用不同的机器学习方法学习出XOR的映射函数.

首先将该任务视为一个回归任务,使用最小均方误差作为损失函数.但是可以看出最小均方误差维持在$0.5$并且很难继续下降.其原因是线性模型难以将样本进行正确划分.

接着引入单隐层的前向神经网络.大多数神经网络单元对输入做仿射变换后,还使用激活函数做非线性变换,因而可以完美的拟合XOR映射函数.在现代的神经网络中,常用\textit{rectified linear unit}(ReLU)作为激活函数.其定义为
\begin{equation}
g(z)=\max\{0,z\}
\end{equation}