\chapter{Linear Factor Models}

许多深度学习学者对输入建立起概率模型$p_{\text{model}}(\bm x)$,借助于该模型使用\textit{概率推断}(probabilistic inference)\footnote{Probabilistic inference is the task of deriving the probability of one or more random variables taking a specific value or set of values. For example, a Bernoulli (Boolean) random variable may describe the event that John has cancer. Such a variable could take a value of $1$ (John has cancer) or $0$ (John does not have cancer). DeepDive uses probabilistic inference to estimate the probability that the random variable takes value $1$: a probability of $0.78$ would mean that John is $78\%$ likely to have cancer.}可以利用环境中的变量去预测环境中的其他变量.这其中许多模型包含隐变量$\bm h$,即$p_{\text{model}}(\bm x)=\mathbbm E_{\bm h}p_{\text{model}}(\bm x|\bm h)$.包含隐变量的分布式表达可以囊括深度前向网络和RNN的所有优势.

包含隐变量的最简单模型是\textit{线性因子模型}(linear factor model).它可以用来组成其他更加复杂的模型.它的数学表达式为
\begin{equation}\label{eq:linear_factor_model}
\bm x=\bm {Wh+b}+\text{noise}
\end{equation}
其中,$\bm h\sim p(\bm h)$,并且其各个分量间相互独立,有$p(\bm h)=\prod_ip(h_i)$;噪声项通常服从各个分量间相互独立的Gaussian分布.

\section{Probabilistic PCA and Factor Analysis}

各种线性因子模型在本质上都是一样的,都可以用式\ref{eq:linear_factor_model}进行概括.不同之处在于,噪声分布的选择以及在观察$\bm x$之前对模型隐变量$\bm h$的先验分布的选择.

\paragraph{Factor Analysis}

Factor analysis隐变量服从单位方差Gaussian分布
\begin{equation}
\bm h\sim\mathcal N(\bm{h;0,I})
\end{equation}
输入变量在给定$\bm h$的情况下条件独立;噪声分布服从对角方差Gaussian分布,$\psi$.

隐变量被用来捕捉不同输入变量间的依赖关系.
\begin{equation}
\bm x\sim\mathcal N(\bm{x;b,WW^T+\psi})
\end{equation}

\paragraph{Probabilistic PCA}

为了将PCA引入到概率框架中,需要把factor analysis中的条件方差限定为全部相同,因此有
\begin{equation}
\bm x\sim\mathcal N(\bm{x;b,WW^T+\sigma^2I})
\end{equation}
大部分数据变量都可以通过隐变量$\bm h$进行捕捉,只存在很小的\textit{重构错误}(reconstruction error)残差$\sigma^2$.

\section{Independent Component Analysis (ICA)}

ICA建立线性因子模型将原始信号分解为独立成分.它有许多不同的形式,但都要求使用者预先设定好$p(\bm h)$,接着模型就可以利用定式生成$\bm{x=Wh}$.

使用者所选择的$p(\bm h)$尽可能独立,这样就可以还原出尽可能相互独立的因子.此外还需要注意的是,$p(\bm h)$不能服从Gaussian分布,否则$\bm W$就不唯一.

ICA仅仅能够获得$\bm x$与$\bm h$之间的变换关系,它并不清除$p(\bm h)$的表示方式,因此也不能获得基于$p(\bm x)$的分布.

ICA也有一些拓展形式
\begin{itemize}
    \item 非线性生成式模型:使用非线性方程$f$生成观察到的数据;
    \item nonlinear independent components estimation (NICE)
    \item 用来学习特征群(feature groups),寻找群内统计独立的特征但是群间不独立的特征.
\end{itemize}

\section{Slow Feature Analysis}

SFA通过是学信号学习不变特征.相比于单个度量指标,环境中的重要特征变化很缓慢.基于\textit{慢速原则}(slowness principle)可以在损失函数中引入
\begin{equation}
\lambda\sum_tL(f(\bm x^{(t+1)}),f(\bm x^{(t)}))
\end{equation}
其中,$lambda$是超参数;$f$是特征提取函数;$L$是损失函数;$t$指示时刻.

SFA算法可以归纳为如下的优化问题
\begin{equation}\begin{split}
\min_{\theta}&\mathbbm E_t(f(\bm x^{(t+1)})_i-f(\bm x^{(t)})_i)^2\\
s.t.&\mathbbm E_tf(\bm x^{(t)})_i=0\\
&\mathbbm E_t[f(\bm x^{(t)})_i^2]=1
\end{split}\end{equation}
其中,第一个限制条件确保该优化问题有唯一解;第二个限制条件防止病态问题将所有特征坍缩至$0$.对于多特征的情形,还需要添加限制条件
\begin{equation}
\forall i<j,\mathbbm E_t[f(\bm x^{(t+1)})_i,f(\bm x^{(t)})_j]=0
\end{equation}
这条限制条件对不同特征进行解耦合.

SFA常被拓展至非线性领域用来学习非线性特征.

SFA的一大优势是可以很容易预测出它所学习到的特征形式,但同时它表现受限的原因也很不清晰,推测可能是因为速度施加了过强的影响.

\section{Sparse Coding}

稀疏编码是一种线性因子模型,在无监督特征学习和特征提取机制中被广泛研究.稀疏编码假设噪声服从精度为$\beta$的各向同性Gaussian分布.
\begin{equation}
p(\bm{x|h})=\mathbbm N(\bm{x;Wh+b},\frac{1}{\beta}\bm I)
\end{equation}
$p(\bm h)$通常选定为在$0$附近有尖锐峰值的分布,如Laplace分布和Sutdent-t分布.

由于模型不能通过最大似然度进行训练,通常使用交替训练策略进行decoder和encoder的训练.

并不是所有的稀疏编码方式都显式构建$p(\bm h)$和$p(\bm{x|h})$.

稀疏编码与非参数encoder紧密结合,相比于其他参数encoder,可以更好地最小化重构误差和对数先验.另一项优势是它没有泛化误差.

稀疏编码的一项劣势是给定$\bm x$求取$\bm h$的过程是一个迭代过程,因此需要耗费较长训练时间.另一项劣势是非参数encoder进行反向传播操作并不直观.

\section{Manifold Interpretation of PCA}

线性因子模型中的PCA和factor analysis方法可以被视为学习一个流形.

设encoder为
\begin{equation}
\bm h=f(\bm x)=\bm{W^T(x-\bm\mu)}
\end{equation}
重构过程为
\begin{equation}
\hat{\bm x}=g(\bm x)=\bm{b+Vh}
\end{equation}

线性encoder和decoder的选择可以最小化重构误差
\begin{equation}
\mathbbm E[\|\bm{x-\hat x}\|^2]
\end{equation}

本章介绍的线性因子模型可以被拓展至autoencoder网络和深度概率模型,在同样的任务中可以有更强大的功能和模型族的自由度.