%%%%%%%%%%%%%%%%%%%%%%%%%%%%%%%%%%%%%%%%%%%%%%%%%%%
%% LaTeX book template                           %%
%% Author:  Amber Jain (http://amberj.devio.us/) %%
%% License: ISC license                          %%
%%%%%%%%%%%%%%%%%%%%%%%%%%%%%%%%%%%%%%%%%%%%%%%%%%%

\documentclass[a4paper,12pt,UTF8,hyperref]{book}
\usepackage[T1]{fontenc}
\usepackage[utf8]{inputenc}
\usepackage{lmodern}
%%%%%%%%%%%%%%%%%%%%%%%%%%%%%%%%%%%%%%%%%%%%%%%%%%%%%%%%%
% Source: http://en.wikibooks.org/wiki/LaTeX/Hyperlinks %
%%%%%%%%%%%%%%%%%%%%%%%%%%%%%%%%%%%%%%%%%%%%%%%%%%%%%%%%%
\usepackage[colorlinks,linkcolor=red,anchorcolor=blue,citecolor=green]{hyperref}
\usepackage{graphicx}
\usepackage[english]{babel}
\usepackage{ctex}
\usepackage{titlesec}
\usepackage{bm}
\usepackage{amsmath}
\usepackage{bbold}
\usepackage{color}
\usepackage{xcolor}
%%%%%%%%%%%%%%%%%%%%%%%%%%%%%%%%%%%%%%%%%%%%%%%%%%%%%%%%%%%%%%%%%%%%%%%%%%%%%%%%
% 'dedication' environment: To add a dedication paragraph at the start of book %
% Source: http://www.tug.org/pipermail/texhax/2010-June/015184.html            %
%%%%%%%%%%%%%%%%%%%%%%%%%%%%%%%%%%%%%%%%%%%%%%%%%%%%%%%%%%%%%%%%%%%%%%%%%%%%%%%%
\newenvironment{dedication}
{
   \cleardoublepage
   \thispagestyle{empty}
   \vspace*{\stretch{1}}
   \hfill\begin{minipage}[t]{0.66\textwidth}
   \raggedright
}
{
   \end{minipage}
   \vspace*{\stretch{3}}
   \clearpage
}

\titleformat{\chapter}{\LARGE\bfseries}{Chapter \,\thechapter}{1em}{}[\vspace{-1cm}]

%%%%%%%%%%%%%%%%%%%%%%%%%%%%%%%%%%%%%%%%%%%%%%%%
% Chapter quote at the start of chapter        %
% Source: http://tex.stackexchange.com/a/53380 %
%%%%%%%%%%%%%%%%%%%%%%%%%%%%%%%%%%%%%%%%%%%%%%%%
\makeatletter
\renewcommand{\@chapapp}{}% Not necessary...
\newenvironment{chapquote}[2][2em]
  {\setlength{\@tempdima}{#1}%
   \def\chapquote@author{#2}%
   \parshape 1 \@tempdima \dimexpr\textwidth-2\@tempdima\relax%
   \itshape}
  {\par\normalfont\hfill--\ \chapquote@author\hspace*{\@tempdima}\par\bigskip}
\makeatother

%%%%%%%%%%%%%%%%%%%%%%%%%%%%%%%%%%%%%%%%%%%%%%%%%%%
% First page of book which contains 'stuff' like: %
%  - Book title, subtitle                         %
%  - Book author name                             %
%%%%%%%%%%%%%%%%%%%%%%%%%%%%%%%%%%%%%%%%%%%%%%%%%%%

% Book's title and subtitle
\title{\Huge \textit{Deep Learning}\textbf{读书笔记}
% \footnote{This is a footnote.} 
% \\ \huge Sample book subtitle
% \footnote{This is yet another footnote.}
}
% Author
\author{\textsc{黄帅}
% \thanks{\url{www.example.com}}
}


\begin{document}
\frontmatter
\maketitle
\mainmatter
% \begin{dedication}
Be Here Now.
\end{dedication}
\tableofcontents
% \listoffigures
% \listoftables
\chapter*{Preface}

这本册子主要用来记录我在学习\href{http://www.deeplearningbook.org/)}{深度学习}时候的笔记, 以及一些感悟. 可以通过我的博客\href{http://shuaihuang.github.io/Notes/}{在线阅读}或者直接下载\href{https://github.com/ShuaiHuang/DeepLearningNotes/blob/master/Deep-Learning-Reading-Note.pdf}{pdf版文档}阅读。但是\href{https://github.com/ShuaiHuang/DeepLearningNotes/blob/master/Deep-Learning-Reading-Note.pdf}{pdf版文档}读书笔记更新频率会更高一些。

\section*{About Me}

\begin{description}
    \item[GitHub] \href{https://github.com/ShuaiHuang/}{https://github.com/ShuaiHuang/}
    \item[Homepage] \href{http://shuaihuang.github.io/}{http://shuaihuang.github.io/}
    \item[Email] \href{mailto:shuaihuang.sjtu@gmail.com}{shuaihuang.sjtu@gmail.com}
\end{description}
% \section*{Acknowledgements}
\begin{itemize}
\item A special word of thanks goes to Professor Don Knuth\footnote{\url{http://www-cs-faculty.stanford.edu/~uno/}} (for \TeX{}) and Leslie Lamport\footnote{\url{http://www.lamport.org/}} (for \LaTeX{}).
\item I'll also like to thank Gummi\footnote{\url{http://gummi.midnightcoding.org/}} developers and LaTeXila\footnote{\url{http://projects.gnome.org/latexila/}} development team for their awesome \LaTeX{} editors.
\item I'm deeply indebted my parents, colleagues and friends for their support and encouragement.
\end{itemize}
\mbox{}\\
%\mbox{}\\
\noindent Amber Jain \\
\noindent \url{http://amberj.devio.us/}
\chapter{Introduction}


本章从AI发展的角度阐述了深度学习是什么,从哪里来,可以用来解决什么问题。

自人工智能发展之初的符号逻辑时代起,人们就借助于规则去解决一些常规方法所不能解决的问题。虽然符号逻辑可以解决一些简单的问题,但是对于一些复杂的情况,如何用符号逻辑去表示这些情况,却比解决问题本身更加困难。之后发展到机器学习时代,人们直接从原始数据中提取出特征,用特征训练模型来解决问题。这就相当于从符号逻辑时代向前更进了一步。但是如何从原始数据中选择合适的特征是一个充满经验性和技巧性的环节。从这一点出发,就发展出了表示学习(repersentation learning)。表示学习通过学习的手段从原始数据中提取出特征,而非传统的手工选取以及手工加工组合特征的方式。深度学习就是表示学习的一种。

深度学习中的深度是指模型的深度较大,具体可以从两个方面进行阐述:
\begin{itemize}
\item 模型的算法流程执行环节较多
\item 描述各个概念关联的图的深度较大
\end{itemize}

但是并没有一个确切的指标表示达到什么样的标准才叫做深度模型。所以深度模型可以泛指包含大量通过学习而得的函数或者通过学习而得的概念的模型,这里的大量是与传统的机器学习中的函数或者概念相对而言的。

基于以上,机器学习就是:
\begin{itemize}
\item 向AI方向更近一步的方式
\item 一种机器学习的方法
\end{itemize}

\section{Who shoule read this book?}

本书的结构为

\begin{itemize}
\item \textbf{Part I} 一些基本概念
\item \textbf{Part II} 一些基本的深度学习算法,以及相应的训练方法
\item \textbf{Part III} 一些前瞻性的想法,并且这些想法极有可能对未来深度学习的发展起到推动性的作用。
\end{itemize}

由于时间有限,本书的重点将放在Part II,其他的部分快速带过。

\section{Historical Treands in Deep Learning}

本节主要介绍了深度学习的发展历史。
\begin{enumerate}
\item 深度学习有着长久的历史;
\item 深度学习近年来有广泛的应用场景是因为可供使用的学习数据增多了;
\item 深度学习模型的复杂度也在不断增加;
\item 深度学习方法解决的问题复杂度也在不断增加,同时准确率也在不断增加。
\end{enumerate}
\chapter{Linear Algebra}

本书的第二章内容主要介绍了深度学习相关的线性代数和矩阵理论基础知识,其中大部分知识比较简单,可以快速带过,这里重点记录一下自己不太熟悉的\textit{奇异值分解}(Singular Value Decomposition)以及线性代数知识在\textit{主成分分析}(Principal Components Analysis)推导过程中的实际应用\footnote{前面7节内容比较简单,这里略去}。

\setcounter{section}{7}
\section{Singular Value Decomposition}

一般来说,一个方形矩阵可以分解为特征值与特征向量相乘的形式。\textit{特征值分解}从提供了将矩阵分解为相乘形式的另一种角度。

根据特征值的性质,有
\begin{equation}
\mathbf A = \mathbf V\text{diag}(\mathbf\lambda)\mathbf V^{-1}
\end{equation}
其中$\mathbf V$是$\mathbf A$所有特征向量组成的矩阵;$\text{diag}(\mathbf\lambda)$是$\mathbf V$中与特征向对应的特征值组成的对角矩阵。

同样的,有奇异值分解
\begin{equation}\label{eq:svd}
\mathbf A=\mathbf{UDV}^{T}
\end{equation}
其中$\mathbf A$是一个$m\times n$矩阵;$\mathbf U$是一个$m\times m$矩阵;$\mathbf V$是一个$n\times n$矩阵;$\mathbf D$是一个$m\times n$矩阵。$\mathbf U$和$\mathbf V$是正交矩阵,$\mathbf D$是对角矩阵。$\mathbf D$对角线上元素被称为奇异值;$\mathbf U$中的每一列被称为\textit{左奇异值向量};$\mathbf V$中的每一列被称为\textit{右奇异值向量}。

$\mathbf A$的左奇异值向量是$\mathbf {AA}^T$对应的特征向量;$\mathbf A$的右奇异值向量是$\mathbf{A}^T\mathbf A$对应的特征向量;非零奇异值是$\mathbf {AA}^T$或$\mathbf{A}^T\mathbf A$特征值的平方根。

\section{The Moore-Penrose Pesudoinverse}

$\mathbf A$的\textit{Moore-Penrose广义逆}被定义为
\begin{equation}
\mathbf A^+=\lim_{\alpha\to 0}(\mathbf{A}^T\mathbf A+\alpha\mathbf I)^{-1}\mathbf A^T
\end{equation}
但是通常情况下采用如下的方法进行计算
\begin{equation}
\mathbf A^+=\mathbf{VD}^+\mathbf U^T
\end{equation}
其中,$\mathbf{U}$,$\mathbf{D}$,$\mathbf{V}$的含义同式\ref{eq:svd}中的定义。$\mathbf D$的广义逆$\mathbf D^+$求取方法为对$\mathbf D$中非零元素取倒数再转置。

使用广义逆求解$\mathbf {Ax=y}$即$\mathbf{x=A^+y}$。如果$\mathbf A$的列数大于行数,则$\|\mathbf x\|_2$在所有可能解中最小;如果$\mathbf A$的行数大于列数,则$\mathbf{Ax}$非常近似于$\mathbf y$即$\|\mathbf{Ax-y}\|_2$最小。
\chapter{Probability and Information Theory}

\setcounter{section}{8}
\section{Common Probability Distributions}

\subsection{Bernoulli Distribution}
\begin{equation}\begin{split}
&P(x=1)=\phi\\
&P(x=0)=1-\phi
\end{split}\end{equation}

\subsection{Multinoulli Distribution}

Bernoulli分布的扩展,假设离散变量有$k$种不同状态。Bernoulli分布和Multinoulli分布在各自的定义域内可以描述任何分布。

\subsection{Gaussian Distribution}

\begin{equation}
\mathcal N(\bm x;\bm{\mu,\beta^{-1}})=\sqrt{\frac{\text{det}(\bm\beta)}{(2\pi)^n}}\exp\Big(-\frac{1}{2}(\bm{x-\mu})^T\bm\beta(\bm{x-\mu})\Big)
\end{equation}
其中$\bm\beta$是精度矩阵,即协方差矩阵的逆矩阵。

\subsection{Exponential and Laplace Distribution}

指数分布:
\begin{equation}
p(x;\lambda)=\lambda\bm{1}_{x\ge 0}\exp(-\lambda x)
\end{equation}
其中$\bm{1}_{x\ge 0}$是指示函数。

拉普拉斯分布
\begin{equation}
\text{Laplace}(x;\mu,\gamma)=\frac{1}{2\gamma}\exp\Big(-\frac{|x-\mu|}{\gamma}\Big)
\end{equation}

\subsection{The Dirac Distribution and Empirical Distribution}

狄利克雷分布:
\begin{equation}
p(x)=\delta(x-\mu)
\end{equation}

经验分布:
\begin{equation}
\hat p(x)=\frac{1}{m}\sum_{i=1}^m\delta(\bm{x-x^{(i)}})
\end{equation}
其中,$\bm x^{(i)}$是从数据集中采样得到的样本集。

\subsection{Mixtures of Distributions}

\begin{equation}
P(x)=\sum_iP(c=i)P(x|c=i)
\end{equation}

\section{Useful Properties of Common Functions}

logistic sigmoid:
\begin{equation}
\sigma(x)=\frac{1}{1+\exp(-x)}
\end{equation}

softplus:
\begin{equation}
\zeta(x)=\log(1+\exp(x))
\end{equation}

\setcounter{section}{12}
\section{Information Theory}

事件$\mathbf x =\bm x$的\textit{自信息}定义为
\begin{equation}
I(\bm x)=-\log P(\bm x)
\end{equation}

我们通常使用\textit{Shannon entropy}量化\textbf{整个概率分布的不确定性}
\begin{equation}
    H(\mathbf x)=\mathbb E_{\mathbf x\sim P}[I(x)]=\mathbb E_{\mathbf x\sim P}[\log P(x)]
\end{equation}
也就是说,一个概率分布的Shannon熵也就是基于该分布的事件所包含信息的期望值。

使用\textit{Kullback-Leibler divergence}度量两个分布$P(\mathbf x)$和$Q(\mathbf x)$的差异程度
\begin{equation}
D_{\text{KL}}(P\|Q)=\mathbb E_{\mathbf x\sim P}\Big[\log\frac{P(x)}{Q(x)}\Big]=
    \mathbb E_{\mathbf x\sim P}[\log P(x)-\log Q(x)]\end{equation}
需要注意的是$D_{\text{KL}}(P\|Q)\ne D_{\text{KL}}(Q\|P)$.

与\textit{Kullback-Leibler divergence}类似的还有\textit{cross-entropy}
\begin{equation}
    H(P,Q)=H(P)+D_{\text{KL}}(P\|Q)=-\mathbb E_{\mathbf x\sim P}\log Q(x)
\end{equation}
通常情况下,规定$\lim_{x\to 0}\,x\log x=0$.

\chapter{Numerical Computation}

机器学习算法中涉及到大量的计算过程.通常情况下,机器学习算法没有解析解,只有通过迭代优化过程得到次优解.本章重点介绍机器学习的数值计算过程相关的知识点.

\section{Overflow and Unferflow}
数值上溢(Overflow)和下溢(Underflow)是两种不同的取整错误(rounding error).其中,下溢发生在数值接近于$0$的时候.有些函数对于接近于$0$和$0$的数字非常敏感.上溢发生在数值的绝对值非常大的时候,会被近似认为是$\infty$或者$-\infty$.在实际过程中,涉及到机器学习底层算法库开发时,必须考虑上溢和下溢的问题,并且在设计优化方法时进行规避.

\section{Poor Condition}
Conditioning是指一个函数的输入发生微小变化时,其对应的输出变化程度,即对输入变化的敏感程度.Poor Condition会放大计算误差.

\section{Gradient-Based Optimization}
大多数深度学习算法都会涉及到某些种类的优化过程.我们通常沿着目标函数的梯度下降方向搜索全局最小值但是如果参数选取不当的话,会收敛到局部极小值或者鞍点上.

\subsection{Beyond the Gradient: Jacobian and Hessian Matrics}
对于输入和输出都是向量的函数,如果要计算其偏导数,就需要用\textbf{Jacobian矩阵}进行表示.设$f:\mathbb R^m\rightarrow\mathbb R^n$,则有Jacboian矩阵$\mathbf J\in\mathbb R^{m\times n}$.其中,$J_{i,j}=\frac{\partial}{\partial x_j}f(\mathbf x)_i$.
如果求二次偏导,则有\textbf{Hessian矩阵},即
\begin{equation}
\mathbf H(f)(\mathbf x)_{i,j}=\frac{\partial^2}{\partial x_i\partial x_j}f(\mathbf x)
\end{equation}
如果想计算某一个方向上的二阶偏导,则有$\mathbf{d}^T\mathbf{Hd}$其中,$\mathbf d$是一单位向量.特别的,如果$\mathbf d$是$\mathbf H$的特征向量,则$\mathbf d$方向上的二阶偏导就是$\mathbf H$对应于$\mathbf d$的特征值.在任何方向上的二阶偏导都在最大特征值和最小特征值的范围之间.

二阶导数可以告诉我们梯度下降的程度.对$f(\mathbf x)$做二阶Taylor展开,有
\begin{equation}
f(\mathbf x)\approx f(\mathbf x^{(0)})+(\mathbf x-\mathbf x^{(0)})^T\mathbf g+\frac{1}{2}(\mathbf x-\mathbf x^{(0)})^T\mathbf H(\mathbf x-\mathbf x^{(0)})
\end{equation}
其中,$\mathbf g$和$\mathbf H$分别是$f(\mathbf x)$在$\mathbf x^{(0)}$点的梯度和Hessian矩阵.设学习率为$\epsilon$,则新到达的点为$\mathbf x^{(0)}-\epsilon\mathbf g$,在该点上取值为
\begin{equation}
f(\mathbf x^{(0)}-\epsilon\mathbf g)\approx f(\mathbf x^{(0)})-\epsilon\mathbf g^T\mathbf g+\frac{1}{2}\mathbf g^T\mathbf{Hg}
\end{equation}
上式中最后一项不能太大,否则新的数值将会比原值还要大.当$\mathbf g^T\mathbf{Hg}$等于或者小于$0$时,$\epsilon$可以取到很大的值;当$\mathbf g^T\mathbf{Hg}$大于$0$时,有
\begin{equation}
\epsilon^\ast=\frac{\mathbf g^T\mathbf g}{\mathbf g^T\mathbf{Hg}}
\end{equation}

优化算法可以按照一阶偏导和二阶偏导进行分类.只用到梯度的优化算法称为一阶优化算法,涉及到Hessian矩阵的优化算法称为二阶优化算法.深度学习的情况复杂,一般的优化算法无法保证结果,因此必须做出一些限制.通常用的限制为\textbf{Lipschitz continuous},即
\begin{equation}
\forall\mathbf x,\forall\mathbf y,|f(\mathbf x)-f(\mathbf y)|\le\mathcal L\|\mathbf{x-y}\|_2
\end{equation}
其中,$\mathcal L$是\textbf{Lipschitz常数}.

凸优化由于有了更多的限制,因此其优化结果可以保证收敛.因此机器学习中在一些特定条件下使用凸优化算法进行优化.

\section{Constrained Optimization}
在机器学习中通常遇到的优化问题是有限制条件的优化问题.为了解决优化问题,一个简单的方法是在梯度下降时将限制条件考虑进去.另外一种更加成熟的方案就是将限制条件考虑进问题中,构造一个与原问题同解的无限制的优化问题.

\textbf{Karush-Kuhn-Tucker}就提供了一种有限制优化问题的解决方法.有Lagrangian
\begin{equation}
L(\mathbf{x,\lambda,\alpha})=f(\mathbf x)+\sum_i\lambda_ig^{(i)}(\mathbf x)+\sum_j\alpha_jh^{(j)}(\mathbf x)
\end{equation}
其中,$g^{(i)}$和$h^{(j)}$分别是等式限制和不等式限制.则原问题与
\begin{equation}
\min_{\mathbf x}\max_{\mathbf\lambda}\max_{\mathbf{\alpha,\alpha}\ge 0}L(\mathbf{x,\lambda,\alpha})
\end{equation}
同解.
\chapter{Machine Learning Basics}

\section{Learning Algorithms}

Mitchell将机器学习定义为
\begin{quote}
A computer program is said to learn from experience \textit E with respect to some class of tasks \textit T and performance measure \textit P, if its performance at tasks in \textit T, as measured by \textit P, improves with experience \textit E.
\end{quote}
本节分别从上述定义中的\textit{T,P,E}三个角度对机器学习的相关知识点进行介绍.

\subsection{The Task, \textit T}

机器学习所关注的任务是人类很难通过具体规则进行编程实现进行解决的任务.学习过程本身不是任务,它只是我们赋予计算机解决问题能力的过程.可以通过机器学习得到解决的任务有
\begin{itemize}
\item 分类问题
\item 有缺失值的分类问题
\item 回归问题
\item 转译(Transcription):输入非结构化表示的数据,输出离散化文字化的数据.
\item 机器翻译
\item 结构化输出(Structured Output)
\item 异常检测
\item 合成与采样
\item 缺失值处理
\item 去噪
\item 概率密度估计或概率分布函数估计
\end{itemize}

\subsection{The Performance Measure, \textit P}

为了可以量化地衡量机器学习算法的表现,通常根据不同的任务类型设计相应的性能度量方式.最常见的是度量模型的\textit{准确率}(accuracy).与此等价的度量指标还有\textit{错误率}(error rate)和\textit{$0$-$1$损失期望}(expected $0$-$1$ loss).对于概率密度估计任务,通常使用在某些样本上的对数概率均值去衡量.

机器学习算法的表现衡量的是算法泛化能力,通常是通过在测试集上的表现去近似估计.选择机器学习算法的度量方法很有技巧性,度量系统的哪一个方面很难决定.即使知道了度量系统的哪些因素,如何量化这些因素也是一大挑战.

\subsection{The Experience, \textit E}

\textbf{根据学习过程中获取到的经验种类,机器学习可以分为\textit{有监督学习}和\textit{无监督学习}两大类.}并且机器学习所获取到的经验是从整个数据集上获取到的.从数学模型上来看,无监督学习所学到的模型是样本的分布$p(\bm x)$,有监督学习所学到的模型是根据样本$\bm x$求其对应样本标记$\bm y$的分布$p(\bm y|\bm x)$.

有监督学习和无监督学习并没有严格的区分界限.假设有向量$\bm x\in\mathbb R^n$,根据概率链式法则
\begin{equation}
p(\bm x)=\prod_{i=1}^np(x_i|x_1,\cdots,x_{i-1})
\end{equation}
则可以得出,无监督学习可以分解为有监督学习的子问题.同样地,我们可以使用无监督学习的方法学得联合概率分布$p(\bm x, \bm y)$,进而求得$p(\bm y|\bm x)$.
\begin{equation}
p(\bm y|\bm x)=\frac{p(\bm{x,y})}{\sum_{\bm y'}p(\bm x,\bm y')}
\end{equation}

除了有监督学习和无监督学习,还有诸如\textit{多样例学习}(multi-instance learning)和\textit{强化学习}(reinforcement learning)等不同形式的学习算法.

\textbf{绝大部分机器学习算法是从数据集中获取到经验的.}通常使用\textit{设计矩阵}(design matrix)对数据集进行表示,其中设计矩阵的每一行对应一个样本,每一列对应一个特征值.这就要求每一个样本所对应的特征维度必须相同.\footnote{9.7和10涉及到异构特征的组织方式.}对于有监督学习,样本还需要包含对应的标签.需要注意的是,标签可能不只使用一个数字进行表示.

\section{Capacity, Overfitting and Underfitting}

机器学习算法最终是在未知样本上执行的,算法在未知样本上也能够取得良好表现效果的能力被称为\textit{泛化能力}(generalization).通常我们通过计算机器学习算法在测试集上的错误率来评估其泛化能力.\textbf{但是如何能够仅仅通过观察算法在训练集上的表现来估计其在训练集上的表现呢?}

通常可以假设训练集和测试集是由同一个数据分布生成的,即训练集和测试集是独立同分布,这被称为\textit{独立同分布假设}(i.i.d. assumptions).假设有一个随机的分布模型,由此分布模型进行数据采样,分别得到训练集合测试集,则期望训练误差和期望测试误差相等.但是在实际中,模型的参数未定,所以测试误差大于或等于训练误差.因此,决定机器学习算法泛化能力的因素有:
\begin{enumerate}
\item 训练误差足够小;
\item 训练误差和测试误差之间的差异足够小.
\end{enumerate}
上述两点分别对应机器学习的两大挑战:欠拟合(underfitting)和过拟合(overfitting).

\textbf{通过对机器学习算法的\textit{容量}(capacity)\footnote{The ability to fit a wide variety of functions.}进行调整,可以对机器学习算法的过拟合或欠拟合情况进行调整.}调整容量的方法有很多,其中一种是选择机器学习算法的\textit{假设空间}(hypothesis space),即机器学习算法允许选择的函数集合.在容量和实际任务以及训练集规模匹配的情况下,机器学习算法会取得最好的效果.但是这是一个依赖经验和技巧的选择过程.除此之外,给定模型函数,通过改变函数参数达到训练目标的过程被称为模型\textit{表示容量}(representational capacity),通过对模型添加限制形成模型\textit{有效容量}(effective capacity),有效能量将会比表示容量小得多.

 提升机器学习模型泛化能力的理论是一个逐步完善的过程.今天的统计学习理论是对\textit{奥卡姆剃刀}(Occam's razor)的进一步完善.\textbf{统计学习理论提供了量化模型容量的多种方法.}其中最有名的是\textit{VC维}(Vapnik-Chervonenkis dimension)\footnote{被假设空间打散的最大示例集大小.}.通过量化模型的容量,统计学习理论可以量化出训练误差和泛化误差差值的上限.但是在实际中,很少通过这种方式对训练误差和泛化误差的差值进行评估.从整体上看,尽管简单模型有较好的泛化能力,但是还是需要选择足够复杂的模型去降低测试集上的误差\footnote{书中用无参模型的极端例子说明复杂度较高的模型可以有较低的训练误差.}.

产生数据集的最本质分布和通过数据集观察到的真实分布$P(\bm x, y)$之间的误差被称为\textit{贝叶斯误差}(Bayes error)\footnote{开个脑洞:真理与认识之间的差异.}对于无参模型,可以通过增加样本集规模使得泛化能力提升并使错误率趋近贝叶斯误差.但是对于通常的有参数模型,通常可以达到的错误率下限是一个比贝叶斯错误率更高的下限.

\subsection{The No Free Lunch Theorem}

机器学习算法可以从有限规模训练集中得到泛化能力较好的模型,从统计学习理论的角度看来是很违背常理的.机器学习算法仅仅给出一个能够使得我们所关注的集合中绝大多数样本近似正确的规则.

机器学习中的\textit{没有免费午餐定义}(no free lunch theorem)说明了,对于所有可能的\textbf{数据生成分布}做同等重要性考虑,任何一种分类算法在未被观测到的新样本上都有同样的错误率.在实际中,我们通常只针对特定的数据生成分布感兴趣,并且机器学习的目的不是对所有任务寻找通解.

\subsection{Regularization}

\textbf{通常的机器学习算法中都包含了归纳偏好}.机器学习算法所需要关注的不仅仅只是其表示容量,函数的可选种类也非常重要.在假设空间中,机器学习算法被赋予一个偏好,使其对特定的函数有所侧重.

偏好还有的目的之一是为了控制机器学算法的过拟合和欠拟合的程度.一般情况下,偏好以正则化项的形式进行表示.偏好还有一个目的是控制模型假设空间的容量.\textbf{正则化是表达机器学习算法偏好的所有方法统称}.

NFL定理从本质上表明没有普适的正则化标准,需要根据不同的任务制定不同的正则化方法.

\section{Hyperparameters and Validation Sets}

机器学习算法通过\textit{超参数}(hyperparameter)控制学习算法的行为,超参数一般是手动调整的.超参数一般是难以通过学习算法进行优化的参数,同时也是难以通过训练集学习得到的参数.需要靠人工经验设定或者网格搜索.

超参数一般通过\textit{验证集}(validation set)进行选取.需要进行区分的是,测试集用来估计泛化误差,不能用来评价超参数的选取或者进行模型选择.

\subsection{Cross-Validation}

当数据集规模较小时,将其进行固定分割会使得测试集规模较小,带来统计误差.交叉验证可以解决这一问题.

\section{Estimators, Bias, and Variance}

统计学可以帮助机器学习从训练集上学得具有较好泛化能力的模型.参数估计,偏差,方差被用来描述模型的泛化能力,欠拟合和过拟合.

\subsection{Point Estimation}

点估计是根据训练样本的分布对兴趣点或者向量提供一个"最好"的预测结果,即
\begin{equation}
\hat{\bm\theta}_m=g(\bm x^{(1)},\cdots,\bm x^{(m)})
\end{equation}
这个定义比较宽泛.一个好的估计可以产生非常接近数据生成的参数$\bm{\bm\theta}$.需要强调的是,这个是频率主义的观点,即认为模型的参数是固定且未知的.

点估计还可以用来估计样本和标记之间的关系,这被称为\textit{函数估计}(Function Estimation).即假设有$y=f(\bm x)+\epsilon$,其中$\epsilon$是$y$不能从$\bm x$中预测到的部分.函数估计是根据模型估计出$f$的估计量$\hat f$.

\subsection{Bias}

偏差的定义为
\begin{equation}
\text{bias}(\hat{\bm\theta}_m)=\mathbb E(\hat{\bm\theta}_m)-{\bm\theta}
\end{equation}
其中${\bm\theta}$是产生数据模型的真实参数;$\hat{\bm\theta}$是估计量.如果有$\text{bias}(\hat{\bm\theta}_m)=0$,则称估计是\textit{无偏的}(unbiased).如果有$\lim_{m\to\infty}\text{bias}(\hat{\bm\theta}_m)=0$,则称估计是\textit{渐进无偏的}(asymptotically unbiased).

通过书中的举例可以看出,无偏估计符合大多数算法的要求,但是他们并不是"最好"的估计.我们通常使用有偏估计去满足算法的其他特性.

\subsection{Variance and Standard Error}

\textit{方差}(variance)反映了数据的散布范围,其平方根被称为\textit{标准差}(standard error).它们分别被定义为$\text{Var}(\hat{\bm\theta})$和$\text{SE}(\hat{\bm\theta})$.同一个分布产生的不同数据集会带来统计量的差异,这种方差的期望值是一个误差来源,如果能量化这个误差来源,对研究机器学习算法也有帮助.

对于标准差,无论是对样本的方差开根号还是对方差的无偏估计开根号,其结果都不是无偏估计,但是它有很重要的用途.在机器学习中,标准差常用来计算平均值的置信区间,并根据置信区间的大小评价算法的好坏.

\subsection{Trading off Bias and Variance to Minimize Mean Squared Error}

偏差和方差描述了两种不同类型的统计量,在实际的机器学习算法中需要对这两者进行适当的权衡.常用的两种方式为:交叉验证和均方误差比较.
\begin{equation}\begin{split}
\text{MSE}&=\mathbb E[(\hat{\bm\theta}_m-{\bm\theta})^2]\\
&=\text{Bias}(\hat{\bm\theta}_m)^2+\text{Var}(\hat{\bm\theta}_m)
\end{split}\end{equation}
理想估计的偏差和方差都很小,直接反映就是均方差小.

偏差和方差与模型的容量有关,模型容量上升会导致偏差下降方差上升.

\subsection{Consistency}\label{subsec:consistency}

通常我们比较关注的情形是,随着训练样本的数量上升,估计量是否趋近于真实值?即
对于$\forall\epsilon>0$,当$m\to0$时,有$P(|\hat{\bm\theta}_m-{\bm\theta}|>\epsilon)\to0$.

一致性保证了样本数量增加的情况下,估计量偏差会逐渐降低.但是反过来不一定成立.

\section{Maximum Likelihood Estimation}

通常使用\textit{极大似然估计}(Maximum Likelihood Estimation)对待估计参数进行估计.假设$p_{\text{model}}(\bm x;{\bm\theta})$是由样本估计出的整体分布,以${\bm\theta}$为参数.其目的就是将$\bm x$映射到真实的分布$p_{\text{data}}(\bm x)$.极大似然度估计定义为
\begin{equation}\begin{split}
{\bm\theta}_{ML}&={\arg\max}_{{\bm\theta}}p_{\text{model}}(\mathbb X;{\bm\theta})\\
&={\arg\max}_{{\bm\theta}}\prod_{i=1}^mp_{\text{model}}(\bm x^{(i)};{\bm\theta})
\end{split}\end{equation}
进行适当的尺度变换和归一化后,可以得到
\begin{equation}
{\bm\theta}_{ML}={\arg\max}_{{\bm\theta}}\mathbb E_{\bm x\sim\hat p_{\text{data}}}\log p_{\text{model}}(\bm x;{\bm\theta})
\end{equation}
其中,$\hat p_{\text{data}}$是样本在训练集上的经验分布(empirical distribution).定性地看,极大似然估计是通过选择${\bm\theta}$使得$p_{\text{model}}$逐步逼近$\hat p_{\text{data}}$.上述两个分布间差异程度可以通过KL散度进行量化:
\begin{equation}
D_{KL}=(\hat p_{\text{data}}\|p_{\text{model}})=\mathbb E_{\bm x\sim\hat p_{\text{data}}}[\log\hat p_{\text{data}}(\bm x)-\log p_{\text{model}}(\bm x)]
\end{equation}
令上式最小,即最小化
\begin{equation}
-\mathbb E_{\bm x\sim\hat p_{\text{data}}}\log p_{\text{model}}(\bm x)
\end{equation}
也就是最小化交叉熵.需要强调的是,任何一个包含负对数似然的损失函数都是训练集经验分布和模型真实分布之间的交叉熵.

实际上,我们可以让模型分布逼近经验分布,但是无法逼近真实分布.

\subsection{Conditional Log-Likelihood and Mean Squared Error}

极大似然估计可以推广到更加一般的条件分布的情形.
\begin{equation}\begin{split}
{\bm\theta}_{ML}&={\arg\max}_{\bm\theta} P(\bm Y|\bm X;{\bm\theta})\\
&={\arg\max}_{\bm\theta}\sum_{i=1}^m\log P(\bm y^{(i)}|\bm x^{(i)};{\bm\theta})
\end{split}\end{equation}
需要注意的是,第二行等式成立的条件是所有样本满足i.i.d条件.

\subsection{Properties of Maximum Likelihood}

如果极大似然估计满足
\begin{enumerate}
\item 真实分布$p_{data}$被包含在模型分布族$p_{model}(\cdot;{\bm\theta})$中;
\item 真实分布$p_{data}$对应的${\bm\theta}$有唯一值.
\end{enumerate}
则估计量满足\ref{subsec:consistency}中提到的一致性.由一致性还可以推导出其他估计量性质,不同的估计量有不同的统计效率(statistic efficiency).最小均方误差(MSE)最为一致性估计具有比较好的性质,被广泛应用于各种算法中,有时还被加上各种归一化项.

\section{Bayesian Statistics}

上节中提到的极大似然估计是频率学派的观点.与之对应的是\textit{贝叶斯学派}(Baysian)认为知识的确定性需要用概率来进行表示\footnote{用概率表征预先知道的信息}.

确定参数${\bm\theta}$的过程首先选取一个不确定性最大即熵最大的分布$p({\bm\theta})$,然后通过观测样本,利用贝叶斯公式求取${\bm\theta}$的后验概率分布.从本质上来说,这是一个熵降低的过程.

贝叶斯估计与极大似然估计有两点不同:
\begin{enumerate}
\item 极大似然估计对${\bm\theta}$进行点估计,而贝叶斯估计根据${\bm\theta}$的全分布去做预测;
\item 贝叶斯估计中的先验分布暗含了参数偏好.
\end{enumerate}

一般情况下,如果训练样本较少,贝叶斯估计会有较好的泛化能力,但是它也存在着计算代价较高的缺点.

\subsection{Maximum A \textit{Posteriori} (MAP) Estimation}

由于贝叶斯统计量估计出的是参数的全分布,但是一般希望导出的是一个点估计.所以使用MAP得到一个对应的点估计.
\begin{equation}\label{eq:map_estimation}
{\bm\theta}_{MAP}={\arg\max}_{\bm\theta} p({\bm\theta}|\bm x)={\arg\max}_{\bm\theta}\log p(\bm x|{\bm\theta})+\log p({\bm\theta})
\end{equation}
由于$p({\bm\theta})$是预先选定的,所以只要关心$p(\bm x|{\bm\theta})$即可.

MAP贝叶斯推断利用了训练集中所不包含的先验知识进行推断,可以减少方差,但同时也会增加偏差.

许多正则化策略在本质上就是MAP贝叶斯推断(在估计过程中添加先验知识$p({\bm\theta})$).并且由于MAP贝叶斯推断提供了一个设计复杂可解释正则化项的方法.

\section{Supervised Learning Algorithms}

本节介绍了几种典型的有监督学习模型.

\subsection{Probabilistic Supervised Learning}

有监督学习可以等价为求取后验概率$p(y|\bm x)$的问题.线性回归问题可以进行扩展用来解决分类任务.在logistic regression中,借助于logistic sigmoid函数,可以将线性回归进行扩展以解决二分类问题.

由于上述方法没有闭式解,所以需要通过使用梯度下降法最小化负对数似然进行迭代求解.这也是其他有监督学习的求解方法.

\subsection{Support Vector Machine}

logistic regression与SVM都被用来解决二分类任务.但是LR给出的是一个概率,SVM直接输出标签.

SVM引入kernel方法后,其性能得到大大提升.核化方法有两大优势:
\begin{enumerate}
\item 将非线性模型进行映射,借助于凸优化求解问题;
\item 比计算原始高维向量更具有优势.
\end{enumerate}

最常用的核函数是\textit{高斯核}(Gaussian kernel),也被称为\textit{径向基函数核}(radial basis function kernel).
\begin{equation}
k(\bm{u,v})=\mathcal N(\bm{u-v};0,\sigma^2\bm I)
\end{equation}
其中$\mathcal N(\bm x;\mu,\Sigma)$是标准正态密度.RBF实质上就是一种模板匹配,算法训练的过程就是如何生成模板.

除了SVM,其他的算法也可以使用核方法.同时核方法的缺点也很明显,损失评估函数是训练样本的线性叠加.但是SVM可以稀疏地叠加部分训练样本.此外,核方法的计算代价也比较高.

\subsection{Other Simple Supervised Learning Algorithms}

$k$-近邻可用于回归任务或者分类任务,其在大规模数据集上可以达到很高的准确率,并趋近于贝叶斯错误率.但是它的计算代价很高,并且不能学习或者选择出区分度高的特征.

\section{Unsupervised Learning Algorithms}

有监督学习和无监督学习之间并没有严格的区分界限.最典型的无监督学习是找到数据"最佳"的表示方式.最佳表示方式一般遵循三条原则:
\begin{itemize}
\item 低维度表示 lower dimensional representations
\item 稀疏表示 sparse representations
\item 独立成分表示 independent representations
\end{itemize}
这三条原则并不是互相排斥的.特征表示是深度学习的一个核心思想,其他的特征表示学习算法也以不同的方式去运用上述三种不同的原则.

\subsection{Principal Component Analysis}

PCA是一种非监督特征学习方法,利用到低维度表示独立成分表示两条原则.PCA只是去除了特征间的线性关系,如果要使得特征间完全独立,还需要去除变量间的非线性关系.PCA在本质上是将原空间中的主成分与新空间的基进行对齐.为进一步去除数据元素间的耦合关系,需要由线性变换向前更进一步.

\subsection{k-means Clustering}

$k$-均值聚类的样本标记可以用one-hot code进行表示,这是一种稀疏表示方式.one-hot code在统计和计算上有很大的优势.

$k$-均值聚类训练过程是一个迭代过程,但它也是一个病态问题,因为没有一个衡量标准去度量聚类结果与真实情况之间的契合程度.实际中可能会存在多种聚类方式从不同方面契合真实情况.

从表示方式上来看,我们更加偏向于\textit{分布式表示}(distributed representations).分布式表示可以减轻算法选择人类喜好特征的负担.

\section{Stochastic Gradient Descent}

\textit{随机梯度下降法}(SGD)是梯度下降法的延伸,对于深度学习的发展起到推动作用.

大规模的训练数据集会带来比较强的泛化性能,但同时也会增加计算代价.一般的损失函数会计算所有样本在损失函数下的和,梯度下降法会在损失函数上计算梯度.

SGD将梯度视为一个期望,即可以通过样本子集的梯度加和估计整体的梯度.

梯度下降对于非凸问题是低效并且是不可靠的,但它可以让损失函数在极短时间内下降,从而使得模型可用.在深度学习领域更是需要借助梯度下降法快速获得可用的模型.

随机梯度下降法还被用于深度学习以外的领域.当训练集样本数量上升时,SGD会逐渐收敛,但其复杂度被认为是$O(1)$.

\section{Building a Machine Learning Algorithm}

深度学习算法的组成元素很简单:数据集的特征组合方式,代价函数,优化方法和模型.通过替换各个元素,可以组合出多种算法.对于非监督学习,这个方法同样适用.

对于代价函数,只要可以计算损失函数的梯度,不计算损失函数也可以进行优化.

大部分机器学习算法都可以从上述几个部分进行归纳\footnote{TODO: 抽空总结一下},只不过有些算法的组成部分并不明显.

\section{Challenges Motivating Deep Learning}

传统算法在AI领域内处理高维数据的能力和泛化能力都存在缺陷,深度学习有效地克服了这一问题.

\subsection{The Curse of Dimensionality}

涉及到高维特征的机器学习算法会变得异常困难,这被称为\textit{维度诅咒}(curse of dimensionality).

\subsection{Local Constancy and Smoothness Regularization}

通过先验知识和学习所得到的分布函数决定了模型的泛化能力.分布函数可以通过显式或者隐式的方式进行选择.其中,隐式地选择方式有:\textit{平滑先验}(smoothness prior)或者\textit{局部一致性先验}(local constancy prior).但是仅仅依靠上述的先验而排除掉其他类型的先验是不够的.

平滑先验和非参数学习算法可以再样本分布符合一定条件的情况下取得较好的泛化能力.但是在高维度情况下,平滑的分布函数可能在某个维度上变得不平滑.如果分布函数比较复杂,这个结论是否还能够成立?答案是肯定的.在限定了数据分布的情况下,平滑假设依然成立.

深度学习算法通过提供不同任务上的通用合理假设以利用这些便利条件.另外一种方法是针对特定任务设定不同的假设,但是这些假设不会被嵌入到深度学习模型之中.深度学习的核心思想是假设样本数据是在多种因素在分层次作用生成的,这就使得样本数量和样本分布空间呈指数关系.

\subsection{Manifold Learning}

在机器学习领域中,流形用以指代小范围内的点集组成的连通域.在流形上,每一个点的维度都有可能不一样.

许多机器学习的任务在全局内学习分布函数的代价很大,流形学习通过添加空间限制克服了这一困难.由于流形是嵌入在高维空间内的,为何不直接使用低维空间或者高维对数据进行表示有两方面的原因:
\begin{enumerate}
\item AI领域内的任务中,图像,文字和声音的分布相对集中,同时噪声在整个空间上的分布很分散,均匀.
\item 这些样本的领域和变换方式很直观,虽然可能不严谨.
\end{enumerate}

使用流形的坐标轴去度量流形具有很深刻的现实意义,并且对于提升机器学习算法很有帮助.但同时也具有很大的挑战性.
\chapter{Deep Feedforward Networks}

\textit{深度前向神经网络}(Deep Feedforward Network)是深度神经网络的经典模型,它通过学习参数$\theta$来估计映射函数$y=f(\mathbf x)$.在前向神经网络中,信息向前流动没有反馈,有反馈的网络被称为\textit{recurrent neural network}(RNN).从工程角度看,前向神经网络是许多工业用途的神经网络的基础,因而它十分重要.前向深度神经网络通过有向无环图直观表示如何将不同的函数进行组合,进而形成网络.

在发展之初,神经网络是一个神经科学模型,但如今它在数学和工程领域内得以发展,也就脱离了原始的神经科学的范畴.

如果想要理解神经网络,最好从线性模型及其不足之处开始,逐步过渡到神经网络领域中.通常如果要将线性模型扩展到非线性领域,常用映射$\phi$为原始数据提供一种新的表示方法.常见的映射有三种形式:
\begin{enumerate}
\item 使用通用的映射$\phi$(如RBF)
\item 人工设计映射$\phi$
\item 使用机器学习的方法学习出映射$\phi$
\end{enumerate}

使用前向神经网络以外的其他模型学习特征的基本准则是深度学习的一项重要内容.

\section{Example: Learning XOR}

本节通过举例使用不同的机器学习方法学习出XOR的映射函数.

首先将该任务视为一个回归任务,使用最小均方误差作为损失函数.但是可以看出最小均方误差维持在$0.5$并且很难继续下降.其原因是线性模型难以将样本进行正确划分.

接着引入单隐层的前向神经网络.大多数神经网络单元对输入做仿射变换后,还使用激活函数做非线性变换,因而可以完美的拟合XOR映射函数.在现代的神经网络中,常用\textit{rectified linear unit}(ReLU)作为激活函数.其定义为
\begin{equation}
g(z)=\max\{0,z\}
\end{equation}
\chapter{Regularization for Deep Learning}

正则化项主要用来降低测试误差,借以提升模型的泛化能力,是机器学习领域内的一个重要研究方向.本章的正则化特指为了降低泛化误差而对学习算法进行的修改.正则化项有两种作用方式:对先验知识进行编码;降低模型复杂度.深度学习中使用的正则化策略绝大部分是基于正则化估计量的策略.学习算法训练出的模型可能会出现
\begin{enumerate}
    \item 将真实数据生成过程排除在模型外;
    \item 符合真实的数据生成过程;\label{enum:2}
    \item 包含了真实的数据生成过程,但是也包含了其他可能的生成过程.\label{enum:3}
\end{enumerate}
正则化的目的就是将上述\ref{enum:3}中的情况转化为\ref{enum:2}的情况.在深度学习的场景中,与真实情况最匹配的模型可能是一个包含正则化项的复杂模型.

\section{Parameter Norm Penalties}

正则化出现的时间比深度学习要早,通过在\textit{目标函数}(objective function)$J$中添加\textit{范数补偿}(norm penalty)$\Omega(\theta)$达到限制模型capacity的目的.正则化后的目标函数为
\begin{equation}\label{eq:cost_func}
\tilde J(\theta;\mathbf{X,y})=J(\theta;\mathbf{X,y})+\alpha\Omega(\theta)
\end{equation}
其中$\alpha\in[0,\infty)$是平衡正则化项权重的超参数.在深度学习中,可以对神经网络的每一层分别设置参数$\alpha$,但是这样计算代价太大,通常将每一层的参数设置成相同的.

\subsection{$L^2$ Parameter Regularization}

$L^2$正则化又被称为\textit{岭回归}(ridge regression)或者\textit{Tikhonov正则化}(Tikhonov regularization).其定义为
\begin{equation}
\tilde J(\mathbf w;\mathbf{X,y})=J(\mathbf w;\mathbf{X,y})+\frac{\alpha}{2}\mathbf w^T\mathbf w
\end{equation}
对应的权重$\mathbf w$更新公式为
\begin{equation}
\mathbf w\leftarrow(1-\epsilon\alpha)\mathbf w-\epsilon\nabla_{\mathbf w}J(\mathbf {w;X,y})
\end{equation}
可以看出正则化对于单步优化的影响为减小了步长.

通过对损失函数在点$\mathbf w^\ast={\arg\min}_\mathbf wJ(\mathbf w)$做二次展开,并求展开式对$\mathbf w$的导数,设在$\tilde{\mathbf w}$导数为0,有
\begin{equation}\begin{split}
\tilde{\mathbf w}&=(\mathbf H+\alpha\mathbf I)^{-1}\mathbf{Hw}^\ast\\
&=\mathbf Q(\mathbf\Lambda+\alpha\mathbf I)^{-1}\mathbf\Lambda\mathbf Q^T\mathbf w^\ast
\end{split}\end{equation}
可以看出,与$\mathbf H$的第$i$个特征向量对应的$\mathbf w^\ast$缩小比例为$\frac{\lambda_i}{\lambda_i+\alpha}$.当$\lambda_i\gg\alpha$时,正则化影响很小;当$\lambda_i\ll\alpha$时,缩小至$0$附近.也就是说,正则化项只有对目标函数下降比较明显的方向上的分量才会做完整的保留.

对线性回归问题,正则化项加上均方误差为
\begin{equation}
\mathbf{(Xw-y)^T(Xw-y)}+\frac{1}{2}\alpha\mathbf{w^Tw}
\end{equation}
对应的对$\mathbf w$导数为$0$的点为
\begin{equation}
\mathbf{w=(X^TX}+\alpha\mathbf{I)^{-1}X^Ty}
\end{equation}
正则化项对输入中方差较高项进行了保留.

\subsection{$L^1$ Regularization}\label{sec:l1_regularization}

$L^1$正则化项被定义为
\begin{equation}
\Omega(\theta)=\|\mathbf w\|_1=\sum_i|w_i|
\end{equation}
相应的目标函数为
\begin{equation}
\tilde J(\mathbf{w;X,y})=\alpha\|\mathbf w\|_1+J(\mathbf{w;X,y})
\end{equation}
正则化项对于梯度的贡献只有符号.为了简化对$1$-范数求导,假设Hessian矩阵是对角矩阵$\mathbf H=diag([H_{1,1},\cdots,H_{n,n}])$,其中$H_{i,i}>0$.有
\begin{equation}
\hat J(\mathbf{w;X,y})=J(\mathbf{w^\ast;X,y})+\sum_i\Big[\frac{1}{2}H_{i,i}(\mathbf{w_i-w_i^\ast})^2+\alpha|w_i|\Big]
\end{equation}
极小值为
\begin{equation}
w_i=\text{sign}(w_i^\ast)\max\{|w_i^\ast|-\frac{\alpha}{H_{i,i}},0\}
\end{equation}
可以看出,$L^1$正则化让解更稀疏.所以它常被用来作特征选择.

此外,$L^2$正则化等价于高斯先验的MAP贝叶斯推断;$L^1$正则化等价于各向同性Laplace分布先验的MAP贝叶斯推断\footnote{TODO:复习与MAP贝叶斯推断的等价关系}.

\section{Norm Penalties as Constrained Optimization}

在前一节中提到的式\ref{eq:cost_func}中的优化问题可以通过Lagrangian函数转化为有限制条件的优化问题
\begin{equation}\label{eq:explicit_constrain}
\mathcal L(\theta,\alpha;\mathbf{X,y})=J(\theta;\mathbf{X,y})+\alpha(\Omega(\theta)-k)
\end{equation}
相应的,解为
\begin{equation}
\theta^\ast={\arg\min}_\theta\max_{\alpha,\alpha\ge 0}\mathcal L(\theta,\alpha)
\end{equation}
定性地分析,最优值$\alpha^\ast$会使$\Omega(\theta)$的值不断减小,但还没有到达让$\Omega(\theta)$小于$k$的程度.因此,我们可以将范数补偿视为在目标函数上增加限制条件.如果不知道限制区域的大小,可以通过调节$\alpha$对限制区域的大小进行调节.

上述显式限制条件有两个优点
\begin{itemize}
    \item 如果知道限制条件,即式\ref{eq:explicit_constrain}中的$k$,那么可以直接对$J(\theta)$使用梯度下降法求下降方向,然后在最小值点附近寻找满足$\Omega(\theta)<k$的点.这样就避免了搜索与$k$相匹配的$\alpha$的过程;
    \item 显式限制条件可以避免因为引入补偿项而导致的非凸优化过程陷入局部极小点.而reprojection方法可以避免这种现象;
    \item 增加优化过程的稳定性.
\end{itemize}

在实际中,常将\textit{列范数限制}(column norm limitation)与显式限制条件相结合使用.

\section{Regularization and Under-Constrained Problems}

在机器学习中,正则化项需要根据实际问题进行妥善定义,因为正则化项需要保证未定问题迭代优化过程的收敛性.

在机器学习以外的领域,也有通过添加正则化项来解决优化问题的方法,如Moore-Penrose广义逆.

\section{Dataset Augmentation}

在机器学习中,可以通过创造"假"数据的方法来增大数据集.这种方法适用于分类问题,但是在其他领域内不太适用,因为可能会改变数据的原始分布.

Dataset augmentation对物体识别问题特别有效.需要注意的是,dataset augmentation不能改变样本的真实标记.

Dataset augmentation在语音识别领域也被广泛应用.

对不同算法进行性能评估比较时,必须将dataset augmentation也考虑进去,在同样的条件下进行比较.

\section{Noise Robustness}

在某些神经网络模型上,给输入添加合适的噪声等价于对权重的范数补偿.这种方法相比于限制权重的大小更加有效,特别是将噪声添加到隐层的输入上.

另外,在网络连接的权重上增加噪声也等价于对模型进行正则化.

增加噪声间接导致了模型学习出稳定的函数.

\subsection{Injection Noise at the Output Targets}

由于某些数据集中部分样本的标注不是完全正确,所以当$y$错误时,直接最大化$\log p(y|\mathbf x)$可能会出现问题.因此\textit{label smoothing}被广泛应用,它使用softmax对模型的0-1硬标记进行替换.

\section{Semi-Supervised Learning}

\textit{半监督学习}是指使用$P(\mathbf x)$中采样得到的未标记样本和从$P(\mathbf{x, y})$中得到的有标记样本共同预测$P(\mathbf{y|x})$的过程.

深度学习中半监督学习的目的是学习特征表示,这样相同类别的样本就会有相似的特征表示.

与传统的有监督学习和无监督学习分离的方法相比,半监督学习通过$P(\mathbf{x,y})$或$P(\mathbf x)$和$P(\mathbf{y|x})$共享参数的方法将两个过程进行融合.

\section{Multi-Task Learning}

多任务学习通过合并不同任务中的样本提升泛化性能.训练出的模型参数分为两类
\begin{itemize}
    \item 任务特定参数
    \item 通用参数
\end{itemize}

由于通用参数的存在,模型的泛化能力得到提升,泛化误差下降.但是多任务学习也有一项前提假设:不同的任务受到同一因素影响,并且这个因素可以通过数据观察得到.

\section{Early Stopping}

早停是指泛化误差达到最小值后,参数更新到达指定迭代次数后终止优化算法,并将参数恢复至泛化误差最小点所对应的参数.通过早停策略可以减少验证集上的误差,从而提升泛化性能.

早停是一种高效的超参数选择方法,它可以选择合适的优化算法迭代次数(超参数),并且不会对原始的算法产生影响.但是同时它也会增加计算代价和存储代价.
\begin{itemize}
    \item \textbf{计算代价}:在优化过程中,每隔一定迭代轮数,需要计算当前模型在验证集上的误差;
    \item \textbf{存储代价}:在优化过程中需要时刻保存一份最优参数.
\end{itemize}

早停策略也可以与其他正则化策略一起使用.

为了充分利用所有的训练数据,可以进行二次训练:
\begin{enumerate}
    \item 重新初始化模型,按照第一轮的超参数进行训练;
    \item 保持第一轮的最优参数,加入新的数据继续进行训练.
\end{enumerate}

从本质上来说,早停将超参数限制在初始值附近的空间中.在一定条件下,早停与$L^2$正则化等价.但是从优化过程上来看,早停自动决定了正则化程度,而$L^2$正则化需要调整超参数.

\section{Parameter Tying and Parameter Sharing}

有时我们需要将正则化之外的先验条件整合到模型中去.此外,我们可以知道模型的结构和相关领域的知识,这样模型之间的参数可能会有一些从属关系(dependency).最常见的模型间的参数从属关系是共享参数.

共享参数的一个重要优势是其存储优势,从而借助于共享参数,可以提升网络的层数.\footnote{TODO: 仔细理解共享参数.CNN是如何实现共享参数的?}

\section{Sparse Representation}

另一种正则化方法是对激活函数添加补偿项使其输出稀疏化.表示方法正则化与参数正则化的机制一致,即
\begin{equation}
\tilde J(\theta;\mathbf{X,y})=J(\theta;\mathbf{X,y})+\alpha\Omega(\mathbf h)
\end{equation}
其中,$\Omega(\mathbf h)$不仅仅可以使用\ref{sec:l1_regularization}节中提到的$L^1$正则补偿项的形式,还可以借助于其他的一些形式实现稀疏化表达.

除了上述添加补偿项实现表示稀疏,还可以通过在激活值上添加硬限制条件实现稀疏表达.典型的方法有\textit{orthogonal matching pursuit}.

从理论上来说,任何有隐层的模型都可以进行稀疏化表示.

\section{Bagging and Other Ensemble Methods}

Bagging是一种集成学习方法,它在不同的训练集上分别训练出不同的模型,对测试样本进行投票得到最终结果,以降低泛化误差.从整体上来看,集成学习器至少和它当中的任何一个学习器表现相当,如果学习器间的误差相互独立,则它的表现将会比其中任何一个学习器都要好.通常学习器会通过不同的学习方法生成,即不同的算法,不同的目标函数以及不同的模型.特别地,通过有放回采样获得$k$个不同的数据子集,在数据子集上分别训练出不同的模型.

由于神经网络参数经过训练后点分布范围比较广泛,所以通常在同一个数据集上进行训练,并将训练完成的模型进行集成.

集成学习是一种极其有效且可靠地降低泛化错误率的方法.但是它的计算和存储开销很大.需要注意的是,并不是所有的集成学习都是为了提升正则化能力,例如Boosting就是为了提升模型的capacity.

\section{Dropout}

Dropout为各种模型提供便于计算的正则化方法.特别地,它为神经网络提供了一种低计算开销的集成方法.具体来说,dropout依次去除原始神经网络中的非输出单元,形成模型子集.子集中模型的数量与神经单元的数量呈指数级关系.为了将模型与dropout进行结合,需要使用基于minibatch的训练方法,以提取出指数级数量的训练子集.

Dropout模型的损失函数定义为
\begin{equation}
\mathbb E_{\mathbf\mu}J(\theta;\mathbf\mu)
\end{equation}
其中,$\mathbf\mu$是\textit{mask vector},用以指定被包含进子模型的神经单元;$J(\theta;\mathbf\mu)$定义了基于模型参数$\theta$和$\mathbf\mu$的损失函数.

Dropout与bagging的不同之处在于
\begin{itemize}
    \item Bagging中学习器间相互独立;而dropout各个子模型之间的参数是共享的;
    \item Bagging中各个学习器经过训练都会达到收敛状态;而dropout大部分子模型都没有被充分地训练(由于参数共享,可以保证模型有较高的泛化能力).
\end{itemize}
\chapter{Optimization for Training Deep Models}

深度学习过程涉及到的优化过程耗时长且有着重要的地位.本章重点介绍深度神经网络训练过程中的优化算法.一般情况下,训练过程中的优化目标是包含正则化项的损失函数.本章将从以下几个方面介绍优化算法
\begin{enumerate}
    \item 机器学习中的优化问题与传统优化问题的区别;
    \item 深度学习中优化问题面临的挑战;
    \item 自适应学习率或损失函数二阶求导;
    \item 通过简单优化组合出复杂优化的策略.
\end{enumerate}

\section{How Learning Differs from Pure Optimization}

对于机器学习算法中的损失函数$J$,其目的是用来提升机器学习效果$P$的,但是对于传统优化问题来说,$J$仅仅只是一个优化目标.通常情况下,损失函数是在所有训练样本上取平均
\begin{equation}
J(\theta)=\mathbb E_{(\mathbf x,y)\sim\hat p_{data}}L(f(\mathbf x;\theta),y)
\end{equation}
但是我们希望得到的是在数据生成模型分布上的期望泛化误差
\begin{equation}\label{eq:expected_generalization_error}
J^\ast(\theta)=\mathbb E_{(\mathbf x,y)\sim p_{data}}L(f(\mathbf x;\theta),y)
\end{equation}

\subsection{Empirical Risk Minimization}

机器学习的目标是降低期望泛化误差,又被称为\textit{风险}(risk),但是通过观察式\ref{eq:expected_generalization_error}可以看出,真实数据分布$p_{data}$未知,因此期望泛化误差不能通过计算直接得到.因此一个替代方案就是最小化\textit{经验风险}(empirical risk)
\begin{equation}
\mathbb E_{(\mathbf x,y)\sim\hat p_{data}}\left[L(f(\mathbf x;\theta),y)\right]=\frac{1}{m}\sum_{i=1}^mL(f(\mathbf x^{(i)};\theta),y^{(i)})
\end{equation}
其中$m$是训练样本的数目.

但是直接使用经验风险最小化策略容易产生过拟合,另外由于$0$-$1$损失函数不能直接求导的原因,深度学习中不使用经验风险最小化策略.

\subsection{Surrogate Loss Functions and Early Stopping}

一般从优化问题效率角度出发使用\textit{代理损失函数}(surrogate loss function)对经验风险函数进行替代.代理损失函数可以延长学习的过程,增强训练出的模型的鲁棒性.同时为了避免过拟合,通常使用早停策略,不会将代理损失函数优化到局部极小值点.

\subsection{Batch and Minibatch Algorithms}

机器学习中的优化问题使用部分样本的损失函数加和去估计整体损失函数,进而对参数进行迭代更新.在实际中,从训练集中采样部分样本评估整体风险的做法很常见.使用采样策略对风险进行评估相比于计算准确的风险,会使得算法收敛速度更快.同时,再重复样本上计算梯度会有大量的冗余.

在全部样本上计算梯度的方法被称为\textit{batch/deterministic gradient method},如果使用单个样本计算梯度的方法被称为\textit{stochastic/online method}.深度学习算法中的优化问题介于两者之间,被称为\textit{minibatch stochastic method}.

minibatch尺寸的选择因素有
\begin{itemize}
    \item 从尺寸增大与回报中做平衡;
    \item 从处理器核的数目做考虑;
    \item 从内存角度做考虑;
    \item $2$的倍数;
    \item 小尺寸有正则化的作用.
\end{itemize}

不同的算法会以不同的形式使用minibatch,并使用其中的不同信息.

需要注意的是,minibatch的随机性很重要.实际中常对样本进行随机打乱来保证随机性.如果样本分解到位,可以使用异步并行的方式更新模型参数.

在样本没有重复的条件下,minibatch方法与梯度下降方法的泛化误差同步下降,但是当样本重复(即开始使用样本进行第二轮迭代)后,偏差会上升.在线学习由于会有数据源源不断地输入进来,所以在泛化误差下降的方面更具有优势.在非在线学习的情形下,第一轮是用样本是无偏梯度估计,之后的迭代是为了减小训练误差与测试误差之间的差异.当训练数据集很大时,每个样本只被使用一次,主要关注的是欠拟合和计算效率问题.

\section{Challenges and Neural Network Optimization}

\subsection{Ill-Conditioning}

在优化问题是凸优化的条件下,也会存在挑战.Hessian矩阵病态条件就是其中之一.如式\ref{eq:second_order_taylor_expansion}中的
\begin{equation}
-\epsilon\mathbf g^T\mathbf g+\frac{1}{2}\epsilon^2\mathbf g^T\mathbf{Hg}
\end{equation}
就有可能存在$\frac{1}{2}\epsilon^2\mathbf g^T\mathbf{Hg}>\epsilon\mathbf g^T\mathbf g$的病态情况.

其他领域内解决病态的问题的方法不能直接用来解决深度学习中的病态问题,需要做一些调整.

\subsection{Local Minima}

在凸优化问题中,局部极小值等价于全局最小值.匪徒问题局部极小并非全局最小,但这并不是主要问题.

如果一个足够大的训练数据集可以训练出唯一一组模型参数,那么就可以说这个模型是\textit{可辨识模型}(model identifiability).神经网络和其他具有隐变量的模型都是不可辨识的.神经网络具有不可辨识性的原因有
\begin{description}
    \item [weight space symmetry]对隐层调神经元不影响最终结果;
    \item [等价缩放]对某一层的神经元做输入输出的等价缩放,不影响最终结果.
\end{description}

不可辨识性导致局部极小值点很多,但是这些局部极小值点有很多具有等价性.

如果局部极小值点的cost function比全局最小值点的cost function数值高,就会有问题.但是如何定量衡量这种差异还没有一个明确的标准.但是现在,研究者们表示,大多数局部极小值点都会一个比较低的cost function数值.

\subsection{Plateaus, Saddle Points and Other Flat Regions}

\textit{鞍点}(saddle point)可以视为代价函数在某个横截面上的极小值点,同时也是另外一个横截面的极大值点.

对于许多随机函数来说,高维空间中的鞍点数目很多.同时,许多随机函数的Hessian矩阵正特征值越多,越有可能达到代价函数的取值越小.这个性质对于神经网络来说同样适用.

鞍点对基于梯度的优化算法影响并不显著,这些算法可以迅速逃离鞍点位置.相比之下,牛顿法容易陷入鞍点.因此,需要使用二阶优化的saddle-free牛顿法.

牛顿法同样容易陷入极大值点以及平坦区域.

\subsection{Cliffs and Exploding Gradients}

梯度优化算法遇到cliff时,步长会变得很长.为了避免这种\textit{gradient clipping}现象,一个简单的解决方案就是使用固定的步长,除非到了极小值附近的区域.

\subsection{Long-Term Dependencies}

当计算图结构较深时,优化算法就遇到了挑战.如在RNN中,相同的操作被重复多次,就出现了\textit{梯度消失和爆炸问题}(vanishing and exploding gradient problem).但是前向神经网络不存在相同操作被重复多次的结构,因此就不会有梯度消失和爆炸问题.

\subsection{Inexact Gradients}

在实际中,通常只能获取到不精确的Hessian矩阵的估计值.当目标函数很难处理时,梯度也很难被精确计算.因此只能通过使用代理函数来避免上述问题.

\subsection{Poor Correspondence between Local and Global Structure}

当梯度的优化方向不正确,前面所描述的问题即使都被妥善解决,也不能提升优化效果.在深度神经网络的训练过程中,大部分时间都被用来选择合适的步长.在实际中,神经网络并不会恰好达到极大值点,极小值点或者鞍点.

大部分研究关注的是如何选取初始值点,以获的较好的优化结果,而不是在优化算法中使用非局部移动策略.

梯度下降和所有的神经网络训练算法都是基于局部移动策略,如果选取正确的初始值点,就可以得到较好的优化效果.

\subsection{Theoretical Limits of Optimization}

针对神经网络设计的优化算法一般都有效果上限.这些优化算法对神经网络的实际应用影响很小\footnote{TODO: 看一下中文版,搞清楚以这一节是什么意思.}.

\section{Basic Algorithms}

\subsection{Stochastic Gradient Descent}

在\textit{随机梯度下降}(stochastic gradient descent)中,很重要的一个参数就是学习率$\epsilon$.实际使用中,这个参数是动态调整的.

因为SGD通过对样本的采样引入了梯度的噪声,即使损失函数达到了最小值点,噪声也不会消失,为了保证SGD能够收敛,学习率$\epsilon$需要满足
\begin{equation}\begin{split}
\sum_{k=1}^\infty\epsilon_k&=\infty\\
\sum_{k=1}^\infty\epsilon_k^2&<\infty
\end{split}\end{equation}

在迭代过程中,$\epsilon$的调整策略为
\begin{equation}
\epsilon_k=(1-\alpha)\epsilon_0+\alpha\epsilon_\tau
\end{equation}
其中$\alpha=\frac{k}{\tau}$.当迭代轮数达到$\tau$之后,$\epsilon$保持常量.

一般通过代价函数的时间曲线选取合适的$\epsilon$值.

SGD的计算时间不会随着样本数量的增加而增加,同时还可以保证算法的收敛性.为了研究优化算法的收敛性,引入度量指标\textit{excess error}
\begin{equation}
J(\mathbf\theta)-\min_{\mathbf\theta}J(\mathbf\theta)
\end{equation}
对于SGD算法,在第$k$轮迭代后,其excess error为$O(\frac{1}{\sqrt{k}})$,当优化问题是凸优化时,excess error为$O(\frac{1}{k})$.除非引入新的前提假设,这个上限是不会被提升的.但是如果收敛速度超过$O(\frac{1}{k})$时,就会导致过拟合.
\chapter{Convolutional Networks}

\textit{卷积神经网络}(convolutional networks)是一种专门用来处理grid-like拓扑关系数据的神经网络.神经网络中至少有一层神经元使用卷积代替矩阵乘法操作.

\section{The Convolution Operation}

最一般形式的卷积的形式为
\begin{equation}
s(t)=\int x(a)w(t-a)da
\end{equation}
其中,$s(t)$被称为\textit{feature map};$x(t)$被称为\textit{input};$w(t)$被称为\textit{kernel}.

在深度神经网络中,常使用离散形式
\begin{equation}
s(t)=(x*w)(t)=\sum_{a=-\infty}^\infty x(a)w(t-a)
\end{equation}
在实际中大多时候是在有限区间上求和.

对于二维空间上的卷积,有
\begin{equation}\begin{split}
S(i,j)=(I*K)(i,j)&=\sum_m\sum_nI(m,n)K(i-m,j-n) \\
&=\sum_m\sum_nI(i-m,j-n)K(m,n)
\end{split}\end{equation}
可见其具有\textit{可交换性}(commutative),这也是为什么进行卷积运算时kernel进行翻转的原因.kernel翻转在数学证明上有很大的便利,但是对于机器学习来说并没有太多的含义.但是在实际中,常根据可交换性,对输入的数据进行翻转而保持kernel不变.

同时需要注意与\textit{cross-correlation}的区别
\begin{equation}
S(i,j)=(I*K)(i,j)=\sum_m\sum_nI(i+m,j+n)K(m,n)
\end{equation}

卷积运算也可以看做是与某个特定矩阵的乘法运算,但是需要注意的是这个特定矩阵是有限制的
\begin{itemize}
    \item 矩阵中的某些项与其他项相等(如\textit{Toeplitz matrix}, \textit{doubly block circulant matrix});
    \item 矩阵是稀疏的,因为与输入图像相比,kernel尺寸是非常小的.
\end{itemize}

\section{Motivation}

在神经网络中引入卷积就引入了\textit{稀疏交互}(sparse interations),\textit{参数共享}(parameter sharing)和\textit{同变性}(equivariant representations).同时,卷积也给神经网络提供了处理可变长输入数据的能力.
\begin{description}
    \item [稀疏交互]
    稀疏交互的原因是因为kernel的尺寸小于输入的尺寸.在神经网络中,稀疏交互相当于\textbf{间接地}与更多的输入进行交互,这样有助于对概念进行抽象.
    \item [参数共享]
    参数共享是指在模型中有多于一个的函数使用相同的参数.在卷积神经网络中,kernel在输入的各个位置上共享.参数共享虽然不能减少计算代价,但是可以降低存储代价.
    \item [同变性]
    参数共享直接导致了对于变换的同变性.如果输入发生变化,那么输出就会以相同的形式进行变化.需要注意的是,卷积并非对于所有的变换都具有同变性,它仍然需要借助于其他的机制去处理不具有同变性的变换.
\end{description}

\section{Pooling}

一层典型的卷积神经网络可以分为三个处理阶段:
\begin{itemize}
    \item 卷积计算
    \item detector stage
    \item pooling function
\end{itemize}
其中pooling用以统计特定区间内的统计量.Pooling有不同的类型\footnote{如何选取不同的类型?各种不同类型pooling的适用范围是什么?}

Pooling对于输入中的微小变换具有近似不变性.本质上来说,Pooling相当于在学习的函数中加入非常强的先验.在实际中,可以在同一个位置pooling不同的卷积结果,这样就可以学习到对于变换具有不变性的特征.

Pooling相当于降维以提高计算效率,同时降低存储开销.

Pooling也是用于处理可变长输入的手段.

Pooling的选取策略有
\begin{itemize}
    \item 不同区域使用不同类型的pooling
    \item 通过学习的方式,在所有区域使用相同类型的pooling
\end{itemize}

Pooling使用top-down信息使得网络结构趋于复杂化.

\section{Convolution and Pooling as Infinitely Strong Prior}

先验可以分为强先验和弱先验两类.弱先验具有较高的熵,强先验具有较低的熵.

卷积神经网络相当于给全连接神经网络加入了极强先验.如果把卷积神经网络当做具有极强先验的全连接神经网络,并在此基础上进行编程实现,将会带来巨大的计算量.但是可以借助于这种观点对卷积神经网络进行数学分析
\begin{itemize}
    \item 卷积神经网络和pooling容易导致欠拟合;
    \item 从学习效果统计量的角度来看,只能用卷积神经网络模型去与其他的卷积神经网络模型作对比.
\end{itemize}
\chapter{Sequence Modeling: Recurrent and Recursive Nets}

\textit{Recurrent neural networks}(RNN)是专门用来处理序列数据的神经网络.它使用共享参数使得模型可以对不同格式(长度)的数据进行处理.RNN还在一维时域做卷积.

\section{Unfolding Computational Graphs}

计算图中的unfolding操作是将迭代操作转换为重复结构非递归操作的过程.RNN有多种形式,任意一种包含递归的函数均可使用RNN来表示.RNN的典型形式为
\begin{equation}
\bm h^{(t)}=f(\bm h^{(t-1)}, \bm x^{(t)};\bm\theta)
\end{equation}
其中$\bm h$是状态变量.由此可见,RNN将$t$时刻任务相关的历史信息经过有损压缩整合成$\bm h^{(t)}$.

RNN模型有两种图的表现形式:第一种是使用一个包含所有成分的结点;第二种是将前一种形式的结点进行展开.

结点展开有两个优点:
\begin{enumerate}
    \item 不管序列的长度有多长,学习到的模型始终有相同的输入长度;
    \item 每一步都可以使用相同的转移函数$f$以及相同的参数.
\end{enumerate}
其中,共享参数可以提升模型的泛化能力.

两种形式的计算图各有用处,并无绝对的优劣之分.

\section{Recurrent Neural Networks}

RNN有三种形式
\begin{itemize}
    \item 隐层单元间有recurrent连接,每一步都会输出结果;
    \item 输出层与隐层单元间有recurrent连接,每一步都会输出结果;
    \item 隐层单元间有recurrent连接,但是只有接受了全部的输入后才会输出结果.
\end{itemize}

在RNN反向传播过程中,损失函数的梯度计算代价很高,不能并行化.此外,正向传播时的状态必须一直记录直到反向传播,因此存储代价也很高.因此,需要考虑替代方案.在此,将对unrolled graph上进行并且有$O(\tau)$复杂度的逆向传播算法称为\textit{back-propagation through time}(BPTT).

\subsection{Teacher Forcing and Networks with Output Recurrence}

仅仅在从前一个时间状态的输出单元到后一个时间状态的隐层单元存在连接的RNN网络的学习能力是非常弱的.因为这种类型的网络缺少从隐层到隐层的周期性连接.为了取得较好的效果,就要求输出必须能够捕捉全面的历史信息.由于损失函数中与时间相关的步骤没有耦合,可以对训练过程进行并行化操作.

\textit{Teacher forcing}操作在每一步都是用上一步的正确输出标记作为当前步骤的输入.引入teacher forcing的目的是为了避免BPTT操作.但是,一旦隐层单元所表示的函数是历史时间状态的函数,就必须进行BPTT计算.

\textit{Teacher forcing}的劣势也很明显,在\textit{open-loop}模式下,当网络的输出作为输入的一部分时,输出反馈的输入在训练环节的分布可能和测试环节时的分布有很大差异.对于这个问题,有两种解决方案:
\begin{itemize}
    \item 模型训练时的输入包含teacher forcing输入和free-running输入;
    \item 随机选择使用生成的数值或者实际数据的数值作为输入,以尽量弥补训练时的输入和测试时的输入.
\end{itemize}

\subsection{Computing the Gradient in a Recurrent Neural Network}

RNN中梯度的计算很直接.需要注意的是,由于参数共享,所以共享参数的梯度是将所有时刻的梯度值进行相加求取.

\subsection{Recurrent Networks as Directed Graphical Models}

对于RNN理论上可以使用任何一种损失函数,最常用的是交叉熵损失函数.

一般地,RNN对应的有向图是一个完备图,任何一对输出节点之间都会有有向依赖关系.通常会对图中交互关系较弱的两个节点之间的边进行简化处理.将隐层单元视为随机变量,这样就可以与历史状态进行解耦合.但是,某些操作仍然会有计算代价,比如缺失值填充和参数优化操作.

RNN中的参数共享意味着$t$与$t+1$时刻的条件分布的参数是固定的.

如果从图中进行采样,RNN必须确定序列的长度.具体方法有
\begin{itemize}
    \item 在原始序列中插入特殊分隔符;
    \item 通过伯努利分布判定序列是否结束;
    \item 将长度$\tau$作为预测目标进行学习预测.
\end{itemize}

\subsection{Modeling Sequences Conditioned on Context with RNNs}

RNN对于条件概率分布有着不同的实现方式.当随机变量$\bm x$是一个固定长度的向量时,有以下方式将其作为附加信息加入RNN模型用以产生$\bm y$序列:
\begin{itemize}
    \item 作为每一个时间步骤的额外输入;
    \item 作为一个初始状态$\bm h^{(0)}$;
    \item 以上两种方法的结合.
\end{itemize}
添加的额外信息$\bm x$可以视为前向神经网络中的bias.

当有多个输入$\bm x^{(t)}$时,可以将$t$时刻的输出作为$t+1$时刻的隐层单元的输入.但是限制条件是,这些序列的长度必须相同.

\section{Bidirectional RNNs}

除了上文提到的当前状态与输入序列中历史值相关的情形,还存在着当前状态与整个输入序列有依赖关系的情形.因此需要引入双向RNN来解决这一问题.在双向RNN中,当前状态对历史状态和未来的状态都存在着依赖关系.

同样地,可以将双向RNN的输入拓展到二维空间,这样就需要在四个方向上分别训练一个RNN.

\section{Encoder-Decoder Sequence-to-Sequence Architectures}

在许多应用情形中,输入序列和输出序列的长度不一定相同.\textit{encoder-decoder}专门用来处理输入输出序列长度不同的情形.其中,encoder用于生成context,decoder用于生成输出序列.context的长度不一定和decoder隐层长度相同.同时encoder-decoder结构还存在一个缺陷,如果context太短的话就不能完整总结较长的输入序列的信息.

\section{Deep Recurrent Networks}

RNN网络中的计算包含3中类型的变换
\begin{itemize}
    \item 从输入到隐状态的变换;
    \item 从前一个隐状态到下一个隐状态的变换;
    \item 从隐状态到输入的变换.
\end{itemize}
RNN中三种操作是一种"浅"变换.为训练出任务所需要的变换,就需要增加深度.

\section{Recursive Neural Networks}

Recursive neural networks与RNN的链式结构不同,它是一种深度树结构.主要被用来处理数据结构.Recursive神经网络中树的深度大大减少,并且可以用来处理长期依赖.数的结构可以独立于数据进行选取.还有外部方法可以用来选择树的结构.除此之外,recursive还有其他的变化类型.

\section{The Challenge of Long-Term Dependencies}

长期交互相比于短期交互,权重水平明显要低.由于RNN中存在着参数共享,所以在状态传递的过程中,特征值会消失或者爆炸.这种情况只存在于RNN网络中,同时也是不可避免的.由此可见,长期依赖是难以学习的.

\section{Echo State Networks}

在RNN中,从输入单元到隐层单元的权重以及隐层单元到隐层单元的权重很难学习.\textit{Echo state networks}(ESN)或者\textit{liquid state machines}通过使用\textit{reservoir computing}机制设定上述上重权重,进而只要学习输出层的权重即可.reservoir computing在思路上类似于核方法,将长度不定的输入序列转换为定长向量,然后借助于线性判别器对目标任务进行解决.

设置权重的策略为通过设置合适的权重,将状态到状态转移函数的Jacobian矩阵的特征值接近于$1$.但是如果引入了非线性单元,许多环节中的梯度会接近于$0$.这种收缩性映射会导致网络遗忘历史状态.

固定权重使得信息可以向前传播但不会爆炸.

ESN设置权重的方法可以用来初始化RNN网络的参数.

\textit{Leaky Units and Other Strategies for Multiple Time Scales}

为处理长期依赖,模型要在不同时间尺度上处理状态.常用的方法有
\begin{itemize}
    \item 添加跳跃连接\textit{skip connection}
    \item 通过\textit{leaky unit}以不同权重整合信号
    \item 去除精细时间尺度上的某些连接
\end{itemize}

\subsection{Adding Skip Connections through Time}

为了获取粗略时间尺度上的信息,可以直接从过去的变量上直接添加连接到当前的变量上.但是这种方式并不能很好地处理所有长期的依赖关系.

\subsection{Leaky Units and a Specturm of Different Time Scales}

另一种方式是在节点上增加线性自连接.增加了线性自连接的隐层单元类似于\textit{running average},
\begin{equation}
\mu^{(t)}\leftarrow \alpha\mu^{(t-1)}+(1-\alpha)v^{(t)}
\end{equation}
其中,$v^{(t)}$是历史状态的累积量,$\mu^{(t-1)}$是当前的状态,$\alpha$是时间常数.时间常数有两种设置方法:第一,固定为常量;第二,设置为参数,通过学习算法进行学习.

这种处理方式方式可以使得信号更加平滑灵活.

\subsection{Removing Connections}

这种方式主要是通过去除短期的连接增加长期的连接,进而使得信息可以流畅地进行传播.

\section{The Long Short-Term Memory and Other Gated RNNs}

在实际中使用频率比较高的RNN网络是\textit{门限RNN}(gated RNNs).门限RNN会构建出导数不会消失或者爆炸的路径.因此连接权重在每一步都会产生变化.同时,门限RNN将会选择合适对历史状态进行清零.

\subsection{LSTM}

LSTM模型的核心在于
\begin{itemize}
    \item 引入自循环构建梯度传播路径,保证梯度可以维持较长时间;
    \item 自适应调节自循环的权重;
    \item 时间范围可以动态调整.
\end{itemize}

在LSTM单元中,有不同的门限
\begin{itemize}
    \item \textit{forget gate}单元
    \begin{equation}
    f_i^{(t)}=\sigma\Big(b_i^f+\sum_jU_{i,j}^fx_j^{(t)}+\sum)_jW_{i,j}^fh_j^{(t-1)}\Big)
    \end{equation}
    \item \textit{external input gate}单元
    \begin{equation}
    g_i^{(t)}=\sigma\Big(b_i^g+\sum_jU_{i,j}^gx_j^{(t)}+\sum)_jW_{i,j}^gh_j^{(t-1)}\Big)
    \end{equation}
    \item \textit{output gate}单元
    \begin{equation}
    q_i^{(t)}=\sigma\Big(b_i^o+\sum_jU_{i,j}^ox_j^{(t)}+\sum)_jW_{i,j}^oh_j^{(t-1)}\Big)
    \end{equation}
\end{itemize}
其中,$\bm x^{(t)}$是当前输入向量;$\bm h^{(t)}$是当前隐层神经元向量;$\bm{b,U,W}$分别是偏置,输入权重,门限的循环权重.

LSTM更容易捕捉到长期依赖.

\subsection{Other Gated RNNs}

与LSTM不同的是,\textit{gated recurrent units}(GRU)用一个门禁同时控制遗忘单元和更新单元.LSTM和GRU在多种任务上都有很好的表现,并且超越其他变型.

\section{Optimization for Long-Term Dependencies}

在优化问题中,二阶导数和一阶导数可能会同时消失.但使用二阶导数优化方法会有诸多缺陷.Mesterov动量法在精心选取初始化参数后,通过一阶方法也有可能达到同样的效果.\footnote{机器学习中,设计一个更容易优化的模型比设计一个性能更好的优化方法更常见.}

\subsection{Clipping Gradients}

梯度下降过快会导致步长很长,进而可能会导致目标函数值上升.因此引入\textit{梯度截断}(clipping the gradient).
\begin{itemize}
    \item 一种方法是在最小批上对参数梯度进行截断(element-wise);
    \item 另一种方法是在参数更新前对梯度范数进行截断(jointly).
\end{itemize}

当梯度的范数超过阈值时,随机选择一步,效果也会很好.

但是不同最小批上的平均范数阶段梯度并不等价于真实梯度的截断.

\subsection{Regularizing to Encourage Information Flow}

解决梯度消失的方法有
\begin{itemize}
    \item 使用LSTM或者其他自循环以及门限机制;
    \item 另一种方法是正则化或者限制参数,以维持信息传递.
\end{itemize}

通常我们希望,
\begin{equation}
(\nabla_{\bm h^{(t)}}L)\frac{\partial\bm h^{(t)}}{\partial\bm h^{(t-1)}}
\end{equation}
与
\begin{equation}
\nabla_{\bm h^{(t)}}L
\end{equation}
的大小相同.因此,引入正则化项
\begin{equation}
\Omega=\sum_t\Big(\frac{\|(\nabla_{\bm h^{(t)}}L)\frac{\partial\bm h^{(t)}}{\partial\bm h^{(t-1)}}\|}{\|\nabla_{\bm h^{(t)}}L\|}-1\Big)^2
\end{equation}
上式中的正则化项比较复杂,还有近似的估计形式.

这种方法的缺点很明显,在数据比较丰富的情况下,有效性不如LSTM.

\section{Explicit Memory}

知识可以分为显式知识和隐式知识两种类型.人工神经网络在记忆显式知识的时候很吃力.愿意在于其缺少\textit{工作记忆}(working memory)系统用来存储样本.为了解决这一问题,\textit{记忆网络}引入记忆单元,可通过寻址机制对记忆单元进行访问.\textit{神经网络图灵机}(neural Turning machine)使用无监督的方式对记忆单元进行访问,它可以同时读或者写多个记忆单元,并且在每个记忆单元中存储的是向量.这与人类的记忆方式比较类似.

在记忆内容被复制后,信息可以被传递而不用关心梯度是否消失或者爆炸.
\chapter{Practical Methodology}

在机器学习模型训练过程中,涉及到的环节有收集更多的信息,增大或者减少模型的capacity,提升模型的估计推断以及调试软件.在不同的阶段需要做出正确的选择而非盲目猜测.除此之外,还应正确使用常见的算法.实际过程可以归纳为
\begin{enumerate}
    \item 确定任务目标;
    \item 建立端到端流程;
    \item 调试模型确定结果瓶颈;
    \item 对模型做出增量改变.
\end{enumerate}

\section{Performance Metrics}

错误度量指标决定了模型训练的后续动作.首先需要明确的是不可能达到$0$错误率,错误率的下限是贝叶斯错误率.

训练数据集的规模通常也受到限制.

那么基于上述两点,如何能够预估出模型的合理表现?与之相关联的是度量指标的选择.最常见的度量指标是错误率,但是许多应用要求使用更加复杂的度量指标,如\textit{精准率},\textit{召回率},以及基于精准率和召回率的\textit{F-score}.

在一些应用中,需要模型对某些样本做出拒绝判定.这是就要引入\textit{覆盖率}(coverage)的度量指标.即模型可以做出判别的样本占全部样本的比例.

\section{Default Baseline Models}

接下来的步骤是建立端到端的系统.在根据任务内容判定是否使用AI模型后,首先根据数据集的结构选择模型类别;其次选择合适的优化算法,通常使用基于动量的SGD算法或者Adam算法;最后选择正则化方式.

除了使用上述步骤,还可以从其他类似任务建立的模型中进行借鉴.

对于是否使用无监督学习的方式,需要根据问题领域进行细分.

\section{Determining Whether to Gather More Data}

通常会优先考虑收集更多的数据而非改进学习算法.

首先需要根据训练集错误率判断是否需要增加样本;接着从测试集表现判断是否需要收集更多的样本;在确定了需要收集更多样本的结论后,最后需要确定需要收集多少样本.

如果收集样本的代价很高,就需要改善学习算法来降低泛化错误率.

\section{Selecting Hyperparamters}

超参数有两种选择方式:手工选择与自动选择.

\subsection{Manual Hyperparameter Tuning}

手动设定超参数需要对超参数与训练错误率,泛化误差,计算代价的关系有深刻的了解.

手动设定超参数的目的就在运行时间以及内存允许的前提下达到最低泛化误差.除此之外,手动设定超参数还有一个目的是为了使得模型的capacity与任务的复杂度相匹配.

一般来说,泛化错误率高有两种原因:第一种是因为模型欠拟合;第二种是由于训练误差和测试误差之间的差距较大.

按照超参数调节的优先级来说,学习率调节的优先级最高.调节学习率以外的参数要同时监控训练集和测试集.

其他的大多数参数是根据它们增大或者减少模型的capacity来设定的.

正则化只是降低测试误差的一种方式,还有其他的方式可以降低测试误差.

\subsection{Automatic Hyperparameter Optimization Algorithm}

人工设定超参数的前提是人工设定的其实范围比较接近最优解,但是这个前提并不是一直都成立.因此从理论上来说,可以设计一个优化算法自动选择超参数.但是这样就会引入新的超参数.但是新的超参数相比于原有的超参数更加容易设定.

\subsection{Grid Search}

当超参数的数量小于或等于$3$个时,通常使用网格搜索方法设定超参数.

超参数搜索的范围需要根据先验知识进行设定.

网格搜索有一个很明显的缺点就是其计算代价随着参数数目的增长呈指数级增长.

\subsection{Random Search}

相比于网格搜索,随机搜索更加易于实现,方便部署,收敛速度快.

不同于网格搜索,随机搜索中的超参数不用进行离散化或者二值化操作.

随机搜索效率较高的原因是其不用进行实验性的搜索测试.

\subsection{Model-Based Hyperparameter Optimization}

自动超参数优化算法需要计算梯度,为了避免进行梯度计算,建立验证集错误率模型,通过这个模型评估超参数的泛化误差.

Bayesian超参数优化方法很常见,但并不是一种具有普遍适用性的方法.

基于模型的超参数优化方法需要实验性测试来提取出实验中的有效信息.因此效率并不是特别高.

\section{Debugging Strategies}

机器学习模型很难进行调试,同时也很难确定出现的问题是算法的原因还是软件实现的原因.

由于算法的行为很难进行预估,这一点加剧了调试的难度.如果模型各个子模块具有自适应性,调试就会更加困难.

大多数调试策略避开了上述两项困难.

一些主要的调试策略包括:
\begin{itemize}
    \item 模型动作可视化;
    \item 最差结果可视化;
    \item 通过训练错误率和测试错误率对软件进行推断;(如果训练错误率低但是测试错误率高,那么就很有可能训练算法正常但是模型过拟合;如果训练误差和测试误差都很高,那么就很难判断软件有问题还是欠拟合.)
    \item 在小规模数据集上进行测试;
    \item 将逆传播导数与数值导数进行比对,例如将
    \begin{equation}
    f'(x)=\lim_{\epsilon\to 0}\frac{f(x+\epsilon)-f(x)}{\epsilon}
    \end{equation}
    与
    \begin{description}
        \item [finite difference] \begin{equation}f'(x)\approx\frac{f(x+\epsilon)-f(x)}{\epsilon}\end{equation}
        \item [centered difference] \begin{equation}f'(x)\approx\frac{f(x+\frac{1}{2}\epsilon)-f(x-\frac{1}{2}\epsilon)}{\epsilon}\end{equation}
    \end{description}
    进行对比.此外还有对向量以及对复数的比对.
    \item 监控激活函数和梯度的直方图.
\end{itemize}

深度学习的算法还提供了一些对结果的保证.
\chapter{Applications}

\section{Large-Scale Deep Learning}

但个神经元并不能体现出智能的特性,只有大量神经元组合在一起才能体现出智能的特性.

\subsection{Fast CPU Implementations}

精心调试的CPU会有较高的性能,并且有多种策略可供选择,对运行在CPU上的程序进行优化.

\subsection{GPU Implementations}

绝大部分现代神经网络是在GPU上实现的.GPU具有高度并行性和较大内存带宽,但是相比于CPU,它的时钟频率较低,并且分支能力较弱.人工神经网络有同样的特性,因此适合运行在GPU上.基于此,GPU厂商开发出了\textit{通用GPU}(general purpose GPUs)用以处理GPU编程代码,而不仅仅是图形渲染.

但是GPU编程比较困难,大部分工程人员使用现成的代码进行组合以实现自己的模型,尽量避免编写底层代码.

\subsection{Large-Scale Distributed Implementations}

由于单个机器的计算能力有限,因此我们希望可以将训练过程和推断过程转化成分布式计算.

推断过程的分布式计算很简单,因为每个机器可以分别运行模型的不同部分,处理同一个样本,这又被称为\textit{数据并行}(data parallelism).

多个机器在单个数据点上进行协同计算的过程被称为\textit{模型并行}(model parallelism).

\textit{异步SGD}(asynchronous stochastic gradient descent)可以解决数据并行问题.

\subsection{Model Compression}

模型推断时对内存资源的要求更高.通常使用\textit{模型压缩}(model compression)来获取一个对内存需求更少,运行时间更短的小模型来进行存储和执行.

模型压缩可行的前提是原始模型为防止过拟合将不同的模型进行集成,组成了大模型.

\subsection{Dynamic Structure}

对数据处理系统进行加速的一项通用策略是设计带有在图上具有\textit{动态结构}(dynamic structure)的模型去处理输入数据.在神经网络中根据输入使用不同的子网络进行计算的结构被称为\textit{条件计算}(conditional computation).动态结构是从计算机理论到软件工程领域被广泛使用的一项基本原则.

一种动态结构是层级结构分类器,用以提升计算效率.决策树本身就是典型的动态结构.基于同样的思路,\textit{gater}根据当前输入选择不同的专家网络进行计算.当专家网络数目较少时,使用\textit{hard mixture of experts}进行计算.但是当专家网络数目较多时,组合数目也增多,就需要借助额外的手段去训练专家网络的组合了.

另外一种动态结构是开关形式,隐层神经元根据上下文从不同的输入单元获取输入.

需要注意的是,动态结构大大降低了神经网络的并行化操作,往往很难高效地进行编程实现.

\subsection{Specialized Hardware Implementations of Deep Networks}

神经网络有不同的硬件实现方式:
\begin{itemize}
    \item ASICs: application-specific integrated circuit
    \item digital: based on binary representations of numbers
    \item analog: based on physical implementations of continuous values as voltages or currents
    \item hybird implementations: field programmable gated array
\end{itemize}

深度神经网络专用硬件得以发展的原因有两点:第一,可以在推断时候使用较低的精度;第二,通用处理器(CPU/GPU)计算速率增长放缓,需要通过提升并行特性来提升整体性能.

\section{Computer Vision}

\subsection{Preprocessing}

计算机视觉往往需要很少的预处理步骤.最常见的有像素值归一化和输入图像尺寸归一化.对于训练数据集的预处理有\textit{dataset augmentation}.同时对于训练数据集以及测试数据集进行的预处理主要是为了排除数据扰动对模型的影响.通常需要对于输入数据扰动有明确的描述,并且需要确认去除扰动后对模型训练不会产生影响.

\subsubsection{Contrast Normalization}

通常情况下,绝大多数任务中可以去除的样本间差异就是对比度差异.设图像为$X\in\mathcal R^{r\times c\times 3}$,其中$X_{i,j,1}$代表红色通道中第$i$行第$j$列所对应的像素亮度.整幅图像的对比度为
\begin{equation}
\sqrt{\frac{1}{3rc}\sum_{i=1}^{r}\sum_{j=1}^{c}\sum_{k=1}^{3}(X_{i,j,k}-\overline{X})^2}
\end{equation}
其中,$\overline X$是整幅图像的亮度均值
\begin{equation}
\overline X=\frac{1}{3rc}\sum_{i=1}^{r}\sum_{j=1}^{c}\sum_{k=1}^{3}X_{i,j,k}
\end{equation}

\textit{全局对比度归一化}(Global contrast normalization, GCN)旨在防止图像间的对比度有较大差异.常将图像的减去均值亮度,并将亮度分布归一化.即
\begin{equation}
X'_{i,j,k}=s\frac{X_{i,j,k}-\overline X}{\max\Big\{\epsilon,\sqrt{\lambda+\frac{1}{3rc}\sum_{i=1}^{r}\sum_{j=1}^{c}\sum_{k=1}^{3}(X_{i,j,k}-\overline X)^2}\Big\}}
\end{equation}

使用$L^2$范数而非标准差归一化的原因在于,神经网络对于方向的响应要比对于准确位置的响应要好.

需要注意的是,\textit{球化}(sphering)与GCN并不相同.球化是指将主成分的方差进行缩放达到一致.球化通常又被称为\textit{白化}(whitening).

全局对比度归一化并不能对图像中局部细节进行突出显示,由此催生出了\textit{局部对比度归一化}(local contrast normalization, LCN).局部对比度归一化可以通过可分离式卷积进行快速实现.此外,LCN也可作为神经网络的非线性单元.

在实际使用中,需要特别注意LCN的正则化策略.

\subsubsection{Dataset Augmentation}

除了之前介绍的augmentation方法,数据集augumentation有更加高级的变换方式.

\section{Speech Recognition}

语音识别的目的是将包含语言信息的声音信号$\bm X$转换成文字序列$\bm y$.\textit{自动语音识别}(automatic speech recognition)的模型可以概括为
\begin{equation}
f^\ast_{ASR}(\bm X)=\mathop{\arg\max}_{\bm x}P^\ast(\bm y|\mathbf X=\bm X)
\end{equation}

传统方法使用HMM与GMM结合的策略进行语音识别.随着深度模型的发展,许多方法借助于深度神经网络提取特征,并用提取出的特征训练HMM和GMM模型.之后深度模型继续发展,并打破了HMM-GMM模型十多年没有提升的僵局.

语音识别领域内的一项突破是使用卷积神经网络进行语音识别.它以二维的视角看待声音信号.

另外一项突破就是端到端的深度神经网络直接摒弃了HMM模型的使用.其中,声音世界的音素级别的对齐也是一个重要环节.

\section{Natural Language Processing}

\textit{自然语言处理}(natural language processing)是使用计算机对人类语言进行处理的技术.除了神经网络领域内的通用技术外,还需要引入专业领域的知识.

\subsection{$n$-grams}

\textit{语言模型}(language model)是指\textit{标记}(token)序列的概率分布.标记序列定长为$n$的模型被称为$n$-gram.
\begin{equation}
P(x_1,x_2,\cdots,x_\tau)=P(x_1,\cdots,x_{n-1})\prod_{t=n}^\tau P(x_t|x_{t-n+1},\cdots,x_{t-1})
\end{equation}
其中,根据$n$取值不同,$n$-gram可以分为:\textit{unigram}($n=1$),\textit{bigram}($n=2$)和\textit{bigram}($n=3$).

通常使用最大似然度对$n$-gram进行直接训练.并且训练过程中可以对$n-1$-gram一起进行训练.

在句子起始位置的分布通常使用边缘概率分布.

对于条件概率为$0$的情况,需要引入\textit{平滑}(smoothing)策略.一种方法是对所有的概率分布加入非零常量.另一种策略是\textit{back-off}方法,使用低阶$n$-gram模型代替高阶模型.

对于维度灾难问题,使用最近邻查找的方法进行缓解,但不能从根本上克服这一问题.由此引入\textit{基于类别的语言模型}(class-based language model)根据词汇类别共享词汇间的统计信息.但是这种处理方式在信息表示过程中有信息损失.

\subsection{Neural Language Models}

\textit{网络语言模型}(Neural Language Models, NLM)专门用来解决语言模型建模过程中的维度灾难问题.它根据词汇上下文对统计信息进行共享.在相似的句子间建立起关联.

词汇表示又被称为\textit{词汇嵌入}(word embeddings).NLM模型更加注重词汇嵌入这种分布式表示方法.

其他模型也可以使用分布式表示提升模型的性能.

\subsection{High-Dimensional Outputs}

基于词汇的概率分布计算代价很大.最朴素的做法是从隐单元到输入空间建立起放射变换,再用sotmax进行概率估计.

\subsubsection{Use of a Short List}

为减少朴素做法的计算代价,引入有限词汇集,并进一步区分为高频词汇表与低频词汇表.但是泛化能力受到高频词汇的限制.

\subsubsection{Hierarchical Softmax}

另一种减少计算代价的方法是对词汇进行逐级分类,最终形成一棵树.每个单词的概率可以通过到达该词汇所有路径的概率总和得到.

层级结构可以通过机器学习的方法得到.

在训练和测试过程中都会有计算效率高的优势.

同时,这种方法不能解决在给定上下文的时候,相似单词选取的问题.

相比于\textit{基于采样的方法}(sampling-based method),这种方法在实际中的效果并不是很好.

\subsubsection{Importance Sampling}

一种加速NLM训练过程的策略就是避免计算所有词汇出现在下一个位置的概率.但是可以通过对词汇数据集进行采样,计算子集中的每个词汇的概率.
\begin{equation}\begin{split}
\frac{\partial\log P(y|C)}{\partial\bm\theta}&=\frac{\partial\log\text{softmax}_y(\bm a)}{\partial\bm\theta}\\
&=\frac{\partial a_y}{\partial\bm\theta}-\sum_{i}P(y=i|C)\frac{\partial a_i}{\partial\bm\theta}
\end{split}\end{equation}
上式中后一项需要从模型本身进行采样,仍然需要计算所有单词的概率.因此,一般做法是\textit{importance sampling}从另外一个分布中进行采样,并进行修正.针对上述情形,提出了\textit{biased importance sampling},
\begin{equation}
w_i=\frac{p_{n_i}/q_{n_i}}{\sum_{j=1}^Np_{n_j}/q_{n_j}}
\end{equation}
因此有
\begin{equation}
\sum_{i=1}^{|\mathbb V|}P(i|C)\frac{\partial a_i}{\partial\bm\theta}\approx\frac{1}{m}\sum_{i=1}^mw_i\frac{\partial a_{n_i}}{\partial\bm\theta}
\end{equation}

更一般地,为了加速训练过程,可以使用稀疏输出层.

\subsubsection{Noise-Contrastive Estimation and Ranking Loss}

\textit{Ranking loss}将NLM的输出看做每个单词作为一个分数,并且正确的分数$a_y$排序比其他的分数$a_i$要靠前,对应的损失函数为
\begin{equation}
L=\sum_i\max(0,1-a_y+a_i)
\end{equation}
但是它不能提供条件概率的估计.

\subsection{Combining Neural Language Models with $n$-grams}

$n$-gram模型的一大优势是模型capacity比较高,但是计算量比较低.同时NLM的计算量较高.因此考虑将这两者进行结合.

\subsection{Neural Machine Translation}

机器翻译的任务是读入一种语言的句子,并将其翻译成同等含义的另一种语言的句子.从整体上看,机器翻译系统的一个部分是产生候选翻译选项,另外一个部分是对翻译候选项进行评价.

传统的语言模型仅仅输出自然语言句子的概率,为使其能够满足机器翻译任务的需求,需要将其拓展至条件概率的范畴.

基于MLP模型方法的一大缺陷是语句必须是定长.对于这个问题,可以借助于RNN模型,形成encoder-decoder框架.为了产生基于原始句子的条件概率,这个模型必须能够对原始的句子有一个完整的表示策略.

\subsubsection{Using an Attention Mechanism and Aligning Pieces of Data}

机器翻译的一种高效做法是,一次读入整个句子,然后根据相关的上下文,一次输出一个单词.基于这种思想,提出了\textit{attention}模型,主要包含以下几个主要组成部分
\begin{itemize}
    \item 读入原始数据并转换成分布式表达方法;
    \item 前一步输出的特征向量列表;
    \item 依次对内存中数据进行处理并输出结果的模块.
\end{itemize}

\subsection{Historical Perspective}

略.

\section{Other Applications}

\subsection{Recommender Systems}

推荐系统主要分为两种应用情形:在线广告和商品推荐.推荐系统相关的任务可以归纳为有监督学习:给出用户和商品的相关信息,对用户的兴趣指标进行预测,最终具化为一个回归问题或者分类问题.

常用的一种方法是\textit{协同过滤}(collaborative filtering).可以通过有参数方式和无参数方式分别实现.\textit{双线性预测}(bilinear prediction)实有参数实现方式之一.在具体应用中,双线性预测知识集成学习方法的学习器之一.

除了双线性方法,基于网络实现的协同过滤方法是RBM无向概率模型.

协同过滤方法有一个明显的缺陷就是其冷启动问题.即对于新加入的用户或者商品,算法很容易失效.解决这个问题的方法是引入商品和用户的额外信息.

\subsubsection{Exploration Versus Exploitation}

一方面,推荐系统存在一个问题就是获取到的用户喜好有偏差.对于未给用户进行推荐的商品是无法预测出用户的反应的.另一方面,强化学习可以对任何动作和反馈进行学习,并且强化学习在exploration和exploitation之间取得平衡.决定exploration于exploitation偏好的因素之一是时间尺度.

另外存在一个困难就是如何对比不同的强化学习策略.

\subsection{Knowledge Representation, Reasoning and Question Answering}

\subsubsection{Knowledge, Relations and Question Answering}

在知识表达领域,一个有意思的研究方向就是如何训练出一种表达方式,使得分布式表达能够捕捉到不同实体间的关系.

在数学领域,\textit{二维关系}(binary relation)是描述有序元素对之间关系的集合.

在AI领域,关系指的是由简单句法构成的并且高度结构化语言描述的句子.

为了在神经网络上表示关系并进行推理,首先需要建立训练数据集.通常使用\textit{relational database}和\textit{knowledge base};此外,还需要选择合适的模型族.

知识表达的一种实际应用是\textit{link prediction},但是它的结果很难量化评价.另一种实际应用是\textit{word-sense disambiguation}.
\chapter{Linear Factor Models}

许多深度学习学者对输入建立起概率模型$p_{\text{model}}(\bm x)$,借助于该模型使用\textit{概率推断}(probabilistic inference)\footnote{Probabilistic inference is the task of deriving the probability of one or more random variables taking a specific value or set of values. For example, a Bernoulli (Boolean) random variable may describe the event that John has cancer. Such a variable could take a value of $1$ (John has cancer) or $0$ (John does not have cancer). DeepDive uses probabilistic inference to estimate the probability that the random variable takes value $1$: a probability of $0.78$ would mean that John is $78\%$ likely to have cancer.}可以利用环境中的变量去预测环境中的其他变量.这其中许多模型包含隐变量$\bm h$,即$p_{\text{model}}(\bm x)=\mathbbm E_{\bm h}p_{\text{model}}(\bm x|\bm h)$.包含隐变量的分布式表达可以囊括深度前向网络和RNN的所有优势.

包含隐变量的最简单模型是\textit{线性因子模型}(linear factor model).它可以用来组成其他更加复杂的模型.它的数学表达式为
\begin{equation}\label{eq:linear_factor_model}
\bm x=\bm {Wh+b}+\text{noise}
\end{equation}
其中,$\bm h\sim p(\bm h)$,并且其各个分量间相互独立,有$p(\bm h)=\prod_ip(h_i)$;噪声项通常服从各个分量间相互独立的Gaussian分布.

\section{Probabilistic PCA and Factor Analysis}

各种线性因子模型在本质上都是一样的,都可以用式\ref{eq:linear_factor_model}进行概括.不同之处在于,噪声分布的选择以及在观察$\bm x$之前对模型隐变量$\bm h$的先验分布的选择.

\paragraph{Factor Analysis}

Factor analysis隐变量服从单位方差Gaussian分布
\begin{equation}
\bm h\sim\mathcal N(\bm{h;0,I})
\end{equation}
输入变量在给定$\bm h$的情况下条件独立;噪声分布服从对角方差Gaussian分布,$\psi$.

隐变量被用来捕捉不同输入变量间的依赖关系.
\begin{equation}
\bm x\sim\mathcal N(\bm{x;b,WW^T+\psi})
\end{equation}

\paragraph{Probabilistic PCA}

为了将PCA引入到概率框架中,需要把factor analysis中的条件方差限定为全部相同,因此有
\begin{equation}
\bm x\sim\mathcal N(\bm{x;b,WW^T+\sigma^2I})
\end{equation}
大部分数据变量都可以通过隐变量$\bm h$进行捕捉,只存在很小的\textit{重构错误}(reconstruction error)残差$\sigma^2$.

\section{Independent Component Analysis (ICA)}

ICA建立线性因子模型将原始信号分解为独立成分.它有许多不同的形式,但都要求使用者预先设定好$p(\bm h)$,接着模型就可以利用定式生成$\bm{x=Wh}$.

使用者所选择的$p(\bm h)$尽可能独立,这样就可以还原出尽可能相互独立的因子.此外还需要注意的是,$p(\bm h)$不能服从Gaussian分布,否则$\bm W$就不唯一.

ICA仅仅能够获得$\bm x$与$\bm h$之间的变换关系,它并不清除$p(\bm h)$的表示方式,因此也不能获得基于$p(\bm x)$的分布.

ICA也有一些拓展形式
\begin{itemize}
    \item 非线性生成式模型:使用非线性方程$f$生成观察到的数据;
    \item nonlinear independent components estimation (NICE)
    \item 用来学习特征群(feature groups),寻找群内统计独立的特征但是群间不独立的特征.
\end{itemize}

\section{Slow Feature Analysis}

SFA通过是学信号学习不变特征.相比于单个度量指标,环境中的重要特征变化很缓慢.基于\textit{慢速原则}(slowness principle)可以在损失函数中引入
\begin{equation}
\lambda\sum_tL(f(\bm x^{(t+1)}),f(\bm x^{(t)}))
\end{equation}
其中,$lambda$是超参数;$f$是特征提取函数;$L$是损失函数;$t$指示时刻.

SFA算法可以归纳为如下的优化问题
\begin{equation}\begin{split}
\min_{\theta}&\mathbbm E_t(f(\bm x^{(t+1)})_i-f(\bm x^{(t)})_i)^2\\
s.t.&\mathbbm E_tf(\bm x^{(t)})_i=0\\
&\mathbbm E_t[f(\bm x^{(t)})_i^2]=1
\end{split}\end{equation}
其中,第一个限制条件确保该优化问题有唯一解;第二个限制条件防止病态问题将所有特征坍缩至$0$.对于多特征的情形,还需要添加限制条件
\begin{equation}
\forall i<j,\mathbbm E_t[f(\bm x^{(t+1)})_i,f(\bm x^{(t)})_j]=0
\end{equation}
这条限制条件对不同特征进行解耦合.

SFA常被拓展至非线性领域用来学习非线性特征.

SFA的一大优势是可以很容易预测出它所学习到的特征形式,但同时它表现受限的原因也很不清晰,推测可能是因为速度施加了过强的影响.

\section{Sparse Coding}

稀疏编码是一种线性因子模型,在无监督特征学习和特征提取机制中被广泛研究.稀疏编码假设噪声服从精度为$\beta$的各向同性Gaussian分布.
\begin{equation}
p(\bm{x|h})=\mathbbm N(\bm{x;Wh+b},\frac{1}{\beta}\bm I)
\end{equation}
$p(\bm h)$通常选定为在$0$附近有尖锐峰值的分布,如Laplace分布和Sutdent-t分布.

由于模型不能通过最大似然度进行训练,通常使用交替训练策略进行decoder和encoder的训练.

并不是所有的稀疏编码方式都显式构建$p(\bm h)$和$p(\bm{x|h})$.

稀疏编码与非参数encoder紧密结合,相比于其他参数encoder,可以更好地最小化重构误差和对数先验.另一项优势是它没有泛化误差.

稀疏编码的一项劣势是给定$\bm x$求取$\bm h$的过程是一个迭代过程,因此需要耗费较长训练时间.另一项劣势是非参数encoder进行反向传播操作并不直观.

\section{Manifold Interpretation of PCA}

线性因子模型中的PCA和factor analysis方法可以被视为学习一个流形.

设encoder为
\begin{equation}
\bm h=f(\bm x)=\bm{W^T(x-\bm\mu)}
\end{equation}
重构过程为
\begin{equation}
\hat{\bm x}=g(\bm x)=\bm{b+Vh}
\end{equation}

线性encoder和decoder的选择可以最小化重构误差
\begin{equation}
\mathbbm E[\|\bm{x-\hat x}\|^2]
\end{equation}

本章介绍的线性因子模型可以被拓展至autoencoder网络和深度概率模型,在同样的任务中可以有更强大的功能和模型族的自由度.
\chapter{Autoencoders}

Autoencoder是一个试图将输入复制到输出的神经网络.通常只是将输入和输出限制为大致相似,并且只对输入中有用的部分进行描述.传统上autoencoder是降维和特征学习的方式之一,但是近来它也被视为生成式模型的学习方法.尽管autoencoder与前向神经网络有不同之处,它仍旧可以使用\textit{recirculation}\footnote{TODO: 是什么?}进行训练.

\section{Undercomplete Autoencoders}

\textit{Undercomplete autoencoders}的目的在于获取隐变量$\mathbf h$中有用的特性.因此在模型中对其施加了$\mathbf h$维度要比$\mathbf x$维度低的限制.

学习过程可以被视为最小化损失函数
\begin{equation}
L(\mathbf x,g(f(\mathbf)))
\end{equation}
的过程.当$f,g$都是线性函数时,autoencoder可以被视为张开一个和PCA相同的子空间.

但是当encoder和decoder的capacity过高时,autoencoder可能提取不到有用的信息.

\section{Regularized Autoencoders}

在隐变量维度和输入变量维度相同,以及过完备的情况下,autoencoder学不到输入数据的分布.但是借助于正则化,encoder和decoder可以根据模型数据分布训练autoencoder.正则化项可以刺激模型训练出直接复制这种形式以外的表示方法.

除了上述的antoencoder,几乎所有的包含隐变量以及推断过程的生成式模型都可以被视为一种特殊形式的autoencoder.

\subsubsection{Sparse Autoencoders}

\textit{Sparse autoencoders}是指将autoencoder的训练准则中加入稀疏补偿项$\Omega(\mathbf h)$
\begin{equation}
L(\mathbf x,g(f(\mathbf x)))+\Omega(\mathbf h)
\end{equation}

sparse autoencoders的目的是学习对其他任务有用的特征.但是这种正则化项并没有直观的贝叶斯表示方法.

稀疏补偿项使模型隐变量的分布,它提供了近似训练生成式模型的方法.

\subsection{Denoising Autoencoders}

\textit{Denoising autoencoders}通过改变重构错误项来训练autoencoder,它最小化
\begin{equation}
L(\mathbf x,g(f(\tilde{\mathbf x})))
\end{equation}
其中,$\tilde{\mathbf x}$是被噪声干扰的输入$\mathbf x$.

\subsection{Regularizing by Penalizing Derivatives}

这种情况下的补偿项为
\begin{equation}
\Omega(\mathbf{h,x})=\lambda\sum_i\|\nabla_{\mathbf x}h_i\|^2
\end{equation}
它迫使模型在输入$\mathbf x$有微小变化时,模型输出变动不会太大.

\section{Representation Power, Layer Size and Depth}

Autoencoder还可以有深层结构.它本身就是一种前向网络,继承了前向网络的所有优势.理论上,包含隐层的深度autoencoder可以近似拟合任何一种函数.

\section{Stochastic Encoders and Decoders}

前向网络的损失函数和输出单元同样可以用于autoencoder.

以激进的观点抛开前向神经网络来看,\textit{encoding function}$f(\mathbf x)$可以被一般化为\textit{encoding distribution}$p_{encoder}(\mathbf{h|x})$.

任何一个包含隐变量的模型$p_{encoder}(\mathbf{h|x})=p_{model}(\mathbf{h|x})$可以定义一个stochastic encoder
\begin{equation}
p_{encoder}(\mathbf{h|x})=p_{model}(\mathbf{h|x})
\end{equation}
和一个stochastic decoder
\begin{equation}
p_{decoder}(\mathbf{x|h})=p_{model}(\mathbf{x|h})
\end{equation}

\section{Denoising Autoencoders}

\textit{Denoising autoencoder} (DAE)是一种将受到噪声干扰的数据还原为原始数据的autoencoder.它可以归结为前向神经网络,使用与前向神经网络相同的方法进行训练.

\subsection{Estimating the Score}

\textit{Score mathcing}是极大似然度的替代方法.它是一种对概率分布的一致性估计.它利用的方法是促使模型与数据分布在每个训练数据点上拥有相同的score.DAE一个重要的特性是其训练准则使得autoencoder可以学习出一个向量场($g(f(\mathbf x))-\mathbf x$)用以估计数据分布的score.

训练过程中使用Gaussian噪声和均方误差的autoencoder等价于训练一个RBM模型,该模型拥有Gaussian可见单元.

\section{Learning Manifolds with Autoencoders}

和其他的机器学习模型一样,autoencoder模型假设数据集中在低维度的流形或者流行子集上.但是autoencoder不仅学习函数来拟合流形,还需要去学习流行的结构.

流形学习有两个因素
\begin{itemize}
    \item 学习$\mathbf h$和$\mathbf x$,建立从输入到输出的映射;
    \item 满足限制条件或者正则化补偿项.
\end{itemize}

流行还有一个重要特性就是其切平面.结合上述两点因素可以看出,流形只对重要变量进行表示,此时只需要对流形切平面上的变化做出反映即可.

最原始的流行学习方法是基于最近邻的非参数化方法.它的缺点是当流行不是很平滑时,需要大量的数据对新数据点进行插值拟合,很难对未知的数据进行泛化.但是常见的AI问题在流形上都是不平滑的,不能通过插值法进行泛化.
\chapter{Representation Learning}

特征表示的方式直接决定信息处理任务的难易程度.一般情况下,一个好的特征表示方式可以使得后续学习任务更容易完成.

有监督学习方式训练出的前向神经网络是一种特征学习方法.但是它并没有对中间特征显式地施加影响.对于学习出的特征表示方式,也可以供不同的任务使用.

大多数特征学习任务必须要在保持尽可能多的信息与获取更优异的特性这两者之间进行权衡.

特征学习提供了一种实现无监督学习和半监督学习的实现方式,可以帮助解决无监督学习和半监督学习问题.

\section{Greedy Layer-Wise Unsupervised Pretraining}

\textit{贪婪逐层无监督预训练}(Greedy Layer-Wise Unsupervised Pretraining)是基于单层特征学习算法(如RBM)进行实现的.它使用无监督学习方式规避有监督学习任务中深度网络各层联合训练的困难.贪婪逐层训练是第一种成功应用于全连接深度结构的训练方法.贪婪是指它对结果进行分块优化,而不是进行整体优化.预训练是指其仅仅只是训练过程的第一步,之后还要进行参数的精细化调整.因此,在有监督学习的框架下,它仅仅只是一种正则化方法或者是一种参数初始化策略.通常我们使用的预训练不仅仅是指预训练阶段,还包含之后的有监督学习阶段.

贪婪逐层无监督预训练不仅仅用于深度神经网络模型上,还可以用于其他无监督学习算法的初始化.

\subsection{When and Why Does Unsupervised Pretraining Work?}

由于无监督预训练有时会对任务有帮助,但是在大多数情况下反而会恶化学习任务.因此需要搞清楚它为什么会有效,因此就可以针对特定任务运用它.

无监督预训练包含两个核心思想
\begin{enumerate}
    \item 深度神经网络的初始参数对算法有较强的正则作用;
    \item 学习输入的分布有助于建立从输入到输出的映射.
\end{enumerate}
其中,核心思想中的第一点并没有被完全理解.我们并没有准确认识到初始化参数所反映的特性在有监督学习阶段被保留多少.第二点的认识稍微深入一些,有些对于无监督学习比较有效果的特征可能同样对有监督学习有效果.但是并不清楚哪种任务可以从无监督学习中取得收益.

从特征学习的角度来看,无监督预训练在原始特征表示比较弱的情况下,无监督预训练会更加有效.从正则化角度来看,无监督预训练在有标记样本数量较少的情况下,对提升模型的效果更有帮助.\footnote{其他无监督预处理对提升模型效果有正向作用的例子可以参见P$532$.}

无监督预训练还可以提升分类任务以外其他任务的结果,还可以用来提升优化环节而不仅仅只是正则化环节.

无监督预训练可以帮助深度神经网络初始参数保持在特定区域中,从而使得结果可以更好地收敛.

一个重要问题是无监督预训练如何能够起到正则化作用.其中一个假设就是,预训练过程可以使得学习算法发现与数据生成过程中的暗含因素相关的特征.

无监督预训练的劣势有
\begin{itemize}
    \item 它将显式地将学习过程分为两个阶段;
    \item 正则化强度不能控制;
    \item 两个阶段有各自的超参数,超参数过多,事前不能调节.
\end{itemize}

目前除了自然语言处理领域以外的其他领域很少使用无监督预训练.但是预训练的核心思想已经渗透到有监督预训练中.

\section{Transfer Learning and Domain Adaption}

\textit{迁移学习}(transfer learning)和\textit{领域自适应}(domain adaptation)是指在一种分布$P_1$中学习到的知识被用于提升另外一种分布$P_2$的泛化能力.

在迁移学习中,我们假定分布$P_1$中的部分变量可以被用于学习$P_2$的分布.通常情况下,迁移学习,多任务学习和领域自适应可以通过特征学习的方式进行实现,其中的特征对于不同的分布或者任务均有效,这些特征对应于暗含的因子.然而在某些情况下,不同的任务间共享的不是输入的语义信息而是输出的语义信息.因此需要在较高的逻辑层次进行概念迁移.

领域自适应是指任务相同但是输入的分布有轻微的不同.

另外还有一个相关的问题是\textit{概念漂移}(concept drift),可以将其视为一种随着时间发展对数据分布进行逐渐演进的迁移学习.概念漂移和迁移学习都可以被视为多任务学习的特例.

特征学习的核心思想在于相同的表示方式对于不同的分布均有效.

如果使用更深的模型进行特征表示,那么在新的分布上,其效果会更好.

迁移学习的极端情况是\textit{one-shot learning}和\textit{zero-shot / zero-data learning}.one-shot learning学习有效的原因可能是其在第一阶段就学习到了数据分布中暗含的因素.

zero-shot learning和zero-data learning仅当附加信息在训练过程中被运用时才会生效.zero-shot learning要求任务的描述方式具有一定的泛化能力才会生效.

\section{Semi-Supervised Distangling of Causal Factors}

对于如何评判一种特征表示方式优劣的问题,一个假设是,理想的特征表示方式和底层暗含的数据生成因素相对应,即特征之间不存在耦合关系.除此之外,特征表示还要比较容易进行塑造.对于许多AI任务,上述两项特性是一致的.

如果$\mathbf y$与$\mathbf x$的起因(causal factors)关系密切,那么$p(\mathbf x)$与$p(\mathbf{y|x})$就会有紧密联系.非监督特征表示试图对变量中暗含的因素进行解耦合,这种策略与半监督学习策略很类似.$p(\mathbf x)$与$p(\mathbf{y|x})$就会有紧密联系体现在
\begin{equation}\begin{split}
p(\mathbf {h,x})&=p(\mathbf{x|h})p(\mathbf h)\\
p(\mathbf x)&=\mathbb E_{\mathbf h}p(\mathbf{x|h})
\end{split}\end{equation}
最佳的关于$\mathbf x$的分布是$\mathbf h$揭示出了$\mathbf x$的真实分布.

在现实中,输入变量通常是多种因素共同作用的结果.一个直接的方式就是暴力搜索.但是在绝大多数情况下,暴力搜索方式并不可行,替代策略有
\begin{itemize}
    \item 同时使用有监督学习与无监督学习去学习特征;
    \item 如果使用无监督特征学习策略,就是用更大的特征集.
\end{itemize}

对于无监督学习策略,可以去改变显著性的定义.具体来说,可以使用\textit{生成式对抗网络}(generative adversarial networks)定义显著性.

学习暗含的起因的收益是,如果生成式过程的\textit{果}为$\mathbf x$,\textit{因}为$\mathbf y$,那么$p(\mathbf{x|y})$对$p(\mathbf y)$的变化具有鲁棒性.

\section{Distributed Representation}

\textit{分布式特征表示}(Distributed Representation)的定义为,特征表示的元素间相互可分离.与之相对应的概念为\textit{符号特征表示}(symbolic representation),每一个单个的符号代表一个类别.

对于非分布式特征表示,参数的个数和特征空间中定义的区域的数目是线性关系.

分布式特征表示与符号特征表示的区别在于由于在不同概念间共享属性因而具有泛化能力\footnote{generalization arises due to shared attributes between different concepts.}.分布式特征表示包含有丰富的相似度空间,在这个空间中,语义上相似的概念有较近的距离,这是符号特征表示所不具备的特性.

当使用很复杂的结构可以使用少量参数进行表示时,分布式特征表示具有统计上的优势.传统的符号特征表示泛化能力的前提假设是平滑假设.但是容易受到维度灾难的影响,并且很难根究已有的符号泛化到新的符号.

对于分布式表示,如果有$O(nd)$个参数,那么可以将输入空间划分为$O(n^d)$个区域.分布式表示的capacity仍然受限,因此泛化能力比较强.

\section{Exponential Gains from Depth}

上节提到,深度结构可以提升统计效率.在实际中,深度特征表示可以对低层次特征进行抽象,并进行非线性的加工.

\section{Providing Clues to Discover Underlying Causes}

大多数特征学习的策略是引入有助于发现底层因素的线索.具体有
\begin{itemize}
    \item Smoothness
    \item Linearity
    \item Multiple explanatory factors
    \item Causal factors
    \item Depth, or a hierarchical organization of explanatory factors
    \item Shared factors across tasks
    \item Manifolds
    \item Natural clustering
    \item Temporal and spatial coherences
    \item Sparsity
    \item Simplicity of Factor Dependencies
\end{itemize}
\chapter{Structured Probabilistic Models for Deep Learning}

\textit{结构化概率模型}(Structured Probabilistic Models)是一种使用图的形式描述随机变量间交互关系的方法.

\section{The Challenge of Unstructured Modeling}

深度学习的目标是对机器学习算法进行拓展,以应对为了解决人工智能问题所要面对的挑战.

大部分使用结构化概率模型的任务对输入的结构做一个完整地理解和构建,这些任务包括
\begin{itemize}
    \item 密度估计
    \item 去噪
    \item 缺失值填充
    \item 采样
\end{itemize}

直接使用表格法存储各个随机变量间的交互情况并不可行,原因有
\begin{itemize}
    \item 内存限制
    \item 统计效率
    \item 推断代价
    \item 采样代价
\end{itemize}
但是在实际上,各个变量子集间的影响是间接的.结构化概率模型只关注随机变量间的\textbf{直接关系}.

\section{Using Graphs to Describe Model Structure}

图模型可以分为\textit{有向无环图}与\textit{无向图}两类.

\subsection{Directed Models}

有向图的边是有向的,体现在概率分布上就是条件概率分布.

只要每个变量在图中只有一小部分父节点,那么联合分布就可以使用少量的参数进行表示.

需要注意的是,有向图仅仅对相互间条件独立的变量有简化作用.
\chapter{Monte Carlo Methods}

随机算法可以分为两类:\textit{Las Vegas算法}和\textit{Monte Carlo算法}.Las Vegas算法返回一个精确结果,但是耗费的时间不确定;Monte Carlo算法返回一个近似正确的结果,但是耗费时间比较固定.

\section{Sampling and Monte Carlo Methods}

在机器学习中,通常会从某个概率分布中采样一定的样本,并且使用这些样本去估计某些变量.

\subsection{Why Sampling?}

采样提供了一种灵活的方法,它可以以较低的代价去估计求和以及积分结果.

\subsection{Basics of Monte Carlo Sampling}

\paragraph{核心思想}Monte Carlo采样的核心是将求和或者积分运算视为在某个分布下的期望值运算,并且通过求取平均值计算期望值.
\begin{equation}
s=\sum_{\mathbf x}p(\mathbf x)f(\mathbf x)=E_P\left[f(\mathbf x)\right]
\end{equation}
或者
\begin{equation}
s=\int p(\mathbf x)f(\mathbf x)d\mathbf x=E_P\left[f(\mathbf x)\right]
\end{equation}

一般而言,直接求取分布是比较困难的,因此通常采用的方法是从分布中采样来代替直接求取分布.最后问题可以归结为如何对分布进行采样,从而使得通过采样计算出的期望值近似等于目标和或者目标积分值.

\section{Importance Sampling}

将求和过程或者求积分过程进行分解,其中一个重要环节是确定哪一部分为$p(\mathbf x)$,哪一部分为$f(\mathbf x)$.但是这种分解并没有统一标准,因为分解形式并不唯一.
\begin{equation}
p(\mathbf x)f(\mathbf x)=q(\mathbf x)\frac{p(\mathbf x)f(\mathbf x)}{q(\mathbf x)}
\end{equation}

\paragraph{optimal importance sampling}$p(\mathbf x)f(\mathbf x)$并不一定是最优的分解方式.最优选择$q^\ast$被称为\textit{optimal importance sampling}.

\paragraph{biased importance sampling}另外一种重要的分解方式是\textit{biased importance sampling},它不要求$p$和$q$必须进行归一化.

$q$的选择直接对Monte Carlo估计的效率产生影响,因此进行选择时必须加以注意\footnote{见P$594$页中间部分.}.

尽管importance sampling有许多危险之处,但是它包括深度学习在内的机器学习算法中起到了重要作用.

\section{Markov Chain Monte Carlo Methods}

在许多情况下,我们希望使用Monte Carlo方法,但是很难获取到可用的方法去从$p_{model}$中精确采样或者从一个足够好的分布$q$中进行importance sampling.在无向图中,通常使用无向图表示分布$p_{model}$,因此就可以借助于\textit{Markov Chain}去近似采样.这种方法被称为\textit{Markov Chain Monte Carlo方法}(MCMC方法).

MCMC方法的限制是模型的每一个状态都没有$0$概率.因此该方法适用于基于能量的方法(energy-based method, EBM).

在EBM中进行\textit{ancestral sampling}会遇到先有鸡还是先有蛋的问题,为避免这一问题需要采用Markov链方法.核心思想是状态$\mathbf x$以任意值作为起始状态,不断迭代更新,最终逼近真实的分布\footnote{TODO: 再看一遍参数化的过程.}.这一过程最终收敛到\textit{stationary distribution}或者\textit{equilibrium distribution}.如果状态转移$T$选取正确,那么稳态分布$q$将会与分布$p$相同.

\paragraph{难点} 所有的Markov Chain都包含随机更新过程直至收敛到equilibrium distribution的过程,并从中采样.这一过程被称为\textit{burning in}.这一过程产生的相邻两个样本相互间是耦合的,为了产生不耦合的样本,需要间隔一定的次数进行采样,时间代价会很高.可以使用并行化进行操作.

另外一个难点是事先并不知道经过多少步可以达到稳定状态.其中迭代的步数被称为\textit{mixing time}.$\mathbf A$中比$1$小的特征值决定了mixing time,但是实际中Markov Chain并不能使用矩阵进行标示,因此只能通过估测判定是否mix.

\section{Gibbs Sampling}

确保$q(\mathbf x)$成为可用分布的方法有两种:
\begin{itemize}
    \item 从学习得到的$p_{model}$中获取到$T$;
    \item 直接参数化$T$并对其进行学习.
\end{itemize}

一个从$p_{model}$中进行采样并建立起Markov Chain的方法是使用Gibbs sampling\footnote{具体算法过程可见周志华<机器学习>中7.5节相关内容.}.

\section{The Challenge of Mixing between Separated Modes}

MCMC方法的一个难点是其很有可能mix不充分.这样会导致在高维空间中,MCMC采样会非常耦合.耦合的原因是其在低能量内进行转移,不能顺利逃离低能量区域.这个问题在一定条件下可以通过找到独立成分并进行同步更新加以解决,但是当独立关系很复杂时,这个方法就失效了.

\subsection{Tempering to Mix between Modes}

当一个分布有明显尖峰并且其周围相对平坦时,它很难与其他mode进行mix.基于能量的模型可以通过引入参数$\beta$控制尖峰的分布
\begin{equation}
p_\beta\propto\exp(-\beta E(\mathbf x))
\end{equation}
其中$\beta$常被称为\textit{temperature}的倒数,反映员是基于能量模型的统计物理量.Tempering是一种常用的mixing策略.但是其仍有局限性\footnote{TODO: 再读一遍本节.}.

\subsection{Depth May Help Mixing}

当从包含隐变量的模型$p(\mathbf{h,x})$中进行采样时,如果$p(\mathbf{h|x})$对$\mathbf x$编码过于完善的话,那么$p(\mathbf{x|h})$并不会使$\mathbf x$变化太大,进而使得mixing效果欠佳.一种解决方案是使$\mathbf h$称为深度特征表示.
\chapter{Confronting the Partition Function}\label{ch:partition_function}

在一些情况下,我们需要将$\tilde p$除以\textit{partition function} $Z(\theta)$进行归一化,以获得有效概率分布
\begin{equation}
p(\bm x;\bm\theta)=\frac{1}{Z(\bm\theta)}\tilde p(\bm x;\bm\theta)
\end{equation}
其中partition function是一个对所有未归一化概率的积分(连续变量)或者求和(离散变量)操作.
\begin{equation}\begin{split}
&\int\tilde p(\bm x)d\bm x\\
&\sum_{\bm x}\tilde p(\bm x)
\end{split}\end{equation}

\section{The Log-Likelihood Gradient}

无向图模型学习在最大化对数似然特别困难是因为它的partition function依赖于参数.
\begin{equation}
\nabla_{\bm\theta}\log p(\bm x;\bm\theta)=\nabla_{\bm\theta}\tilde p(\bm x;\bm\theta)-\nabla_{\bm\theta}\log Z(\bm\theta)
\end{equation}
这就是我们所熟知的\textit{positive phase}和\textit{negative phase}分解\footnote{The term positive and negative do not refer to the sign of each term in equation, but rather reflect to their effect on the probability density defined by the model.(\href{http://deeplearning.net/tutorial/rbm.html\#rbm}{http://deeplearning.net})}.

\begin{equation}\label{eq:gradient_of_log_partition_function}
\nabla_{\bm\theta}\log Z=\mathbbm E_{\bm x\sim p(\bm x)}\nabla_{\bm\theta}\log\tilde p(\bm x)
\end{equation}
上述等式是Monte Carolo方法估计极大似然的基础.

在positive phase项中,我们增大从\textbf{数据}中采样的$\bm x$所对应的$\log\tilde p(\bm x)$的值.在negative phase项中,我们通过降低从\textbf{模型}中采样的$\log\tilde p(\bm x)$的值而降低partition function的值\footnote{There are many ways that a learner can capture knowledge about the data generating distribution $p_{data}$ from which training examples were obtained. Some of this encapsulate a model $p_{model}$ of that distribution (or of a derived distribution such as a distribution of output variables given input variables).}.

\section{Stochastic Maximum Likelihood and Contrastive Divergence}

\paragraph{naive approach}式\ref{eq:gradient_of_log_partition_function}的一种直观计算方法为,每次需要计算梯度时都从随机状态burning in一个Markov链.可以看出,这样的方法计算代价较高并且非常低效.

MCMC方法最大化似然度可视为在两种力量间进行平衡:一种力量是提升数据所在的模型分布位置的概率;另外一种力量是压低模型样本出现位置的模型分布概率.

\paragraph{contrastive divergence}contrastive divergence(CD, 或者CD-$K$用以指示$k$步Gibbs的CD算法)在初始化Markov链的每一步都使用从数据分布中采样到的样本.

CD算法容易在supurious modes的情况下失效.并且对于直接训练深度模型没有太大的帮助.

CD算法可以视为对模型进行补偿,当Markov链的输入是从数据分布中采样得到时,其结果变化剧烈.

\paragraph{stochastic maximum likelihood(SML)}另外一种方法是在计算梯度而初始化Markov链的时候,都将前一步的梯度方向考虑进来,用以初始化Markov链.由于可以储存观测变量和隐变量,所以SML可以同时初始化观测变量和隐变量.SML可以高效训练深度模型.

相比于精确采样,CD算法的方差较低,但是SML的方差偏高.

基于MCMC采样的方法适用于几乎所有的MCMC变体.

\paragraph{Fast PCD}一种加速学习过程mixing的方法不是改变Monte Carlo方法,而是改变模型和代价函数的参数化过程.即将参数$\theta$替换为
\begin{equation}
\bm\theta=\bm\theta^{(slow)}+\bm\theta^{(fast)}
\end{equation}
尽管当快速权重可以自由改变时效果才会体现,但是这样就可以使得Markov链进行快速mix.

\section{Pseudolikelihood}

Monte Carlo直接面对partition function的计算过程,与之对应的其他方法则绕考计算partition function的过程,也就是本节所提到的\textit{pseudolikelihood}方法.

Pseudolikelihood方法通过概率的比例部分消除掉partition function的影响.假设将变量$\bm x$分割为$\bm a, \bm b, \bm c$,那么有
\begin{equation}
p(\bm a|\bm b)=\frac{p(\bm{a,b})}{p(\bm b)}=\frac{p(\bm{a,b})}{\sum_{\bm{a,c}}p(\bm {a,b,c})}=\frac{\tilde p(\bm{a,b})}{\sum_{\bm{a,c}}\tilde p(\bm {a,b,c})}
\end{equation}
在实际中,$\bm c$的数目会很多,如果将$\bm c$并入$\bm b$会减少计算量,这样就产生了pseudolikelihood目标函数
\begin{equation}
\sum_{i=1}^n\log p(x_i|\bm x_{-i})
\end{equation}
将计算复杂度向偏差让渡,可得\textit{generalized pseudolikelihood}估计
\begin{equation}
\sum_{i=1}^m\log p(x_{\mathbbm S^{(i)}}|\bm x_{-{\mathbbm S^{(i)}}})
\end{equation}

基于pseudolikelihood的方法表现很大程度上取决于模型如何使用.generalized pseudolikelihood适用的情形为$\mathbbm S$捕捉到了变量间的重要耦合关系.但是它的缺点为,当与其他近似估计方法一起使用时,仅能提供$\tilde p(\bm x)$的下限.

它也不能用于深度模型.但是它可以用于深度网络的单层神经元训练以及不基于下限的近似推断方法.

同时需要注意的是它的计算代价比较大.

\section{Score Matching and Ratio Matching}

\paragraph{Score Matching}在训练模型时可以不用对$Z$及其导数进行估计.$\nabla_{\bm x}\log p(\bm x)$被称为\textit{score}.\textit{score matching}是指
\begin{equation}\begin{split}
L(\bm x,\bm\theta)&=\frac{1}{2}\|\nabla_{\bm x}\log p_{model}(\bm x;\bm\theta)-\nabla_{\bm x}\log p_{data}(\bm x;\bm\theta)\|_2^2\\
J(\bm\theta)&=\frac{1}{2}\mathbbm E_{p_{data}(\bm x)}L(\bm x,\bm\theta)\\
{\bm\theta}^\ast&=\min_{\bm\theta} J(\bm\theta)
\end{split}\end{equation}
同时可以看出,score matching需要知道真实的数据分布$p_{data}$.但是在一定条件下\footnote{见P618},$L(\bm x,\theta)$等价于
\begin{equation}
\tilde L(\bm x,\bm\theta)=\sum_{j=1}^n\Big(\frac{\partial^2}{\partial x_j^2}\log p_{model}(\bm x;\bm\theta)+\frac{1}{2}(\frac{\partial}{\partial x_j}\log p_{model}(\bm x;\bm\theta))^2\Big)
\end{equation}
上式不适用于离散观察变量,但是可以用于离散隐变量.

\paragraph{Generalized Score Matching(GSM)}GSM适用于观察变量是离散的情形,但是不适用于高维离散空间中某些事件概率为$0$的情形.

\paragraph{Ratio Matching} Ratio Matching是特别针对二值数据提出的.
\begin{equation}
L^{(RM)}(\bm x;\bm\theta)=\sum_{j=1}^n\Big(\frac{1}{1+\frac{p_{model}(\bm x;\bm\theta)}{p_{model}(f(\bm x), j;\bm\theta)}}\Big)
\end{equation}

但是ratio matching的计算代价较高.

Ratio Matching适用于高维离散的情形.

\section{Denoising Score Matching}

在一些情况下,我们希望正则化score matching,就需要将真实数据分布$p_{data}$替换为
\begin{equation}
p_{smoothed}(\bm x)=\int p_{data}(\bm y)q(\bm x|\bm y)d\bm y
\end{equation}

如果给定足够的capacity,任何一致性估计都会使得$p_{model}$在训练集上样本上形成Dirac分布.对$q$进行平滑有助于减轻这个问题,但同时也损失了一部分渐进一致性.

\section{Noise-Contrastive Estimation}

\paragraph{Noise-contrastive estimation(NCE)}在NCE方法中,概率分布估计被显式定义为
\begin{equation}\label{eq:noise-contrastive-estimation}
\log p_{model}(\bm x)=\log\tilde p_{model}(\bm x;\bm\theta)+c
\end{equation}
其中,引入的$c$是$-\log Z(\bm\theta)$的近似估计.

如果采用最大似然度对式\ref{eq:noise-contrastive-estimation}进行估计,那么$c$将会被置为一个非常高的值,而不是产生一个有效分布.

\paragraph{Noise Distribution}因此引入噪声分布$p_{model}(\bm x)$,
\begin{equation}\begin{split}
p_{joint}(y=1)&=\frac{1}{2}\\
p_{joint}(\bm x|y=1)&=p_{model}(\bm x)\\
p_{joint}(\bm x|y=0)&=p_{noise}(\bm x)
\end{split}\end{equation}
其中$y$是一个开关变量,决定是从模型分布还是噪声分布中产生$\bm x$.这样就可以使用最大似然度解决上述问题
\begin{equation}
\bm\theta,c=\mathop{\arg\max}_{\bm\theta,c}\mathbbm E_{\bm{x,y}\sim p_{train}}\log p_{joint}(y|\bm x)
\end{equation}

NCE使用最为成功的是没有太多随机变量的情形,但是这些随机变量可以有很数目的数值.同时它的缺陷是$p_{noise}$必须容易估计,并且很容易从过度限制中进行采样.

NCE的核心思想是一个好的生成式模型应该可以区分开数据和噪声.

\section{Estimating the Partition Function}

在模型评估,监控训练效果,以及对比模型的过程中,必须计算partition function.

如果我们需要计算$\mathcal M_A$或者$\mathcal M_B$的真实分布,并且知道两个分布的partition function的比值,那么只需要计算其中一个partition function就可以了.
\begin{equation}
Z(\bm\theta_B)=rZ(\bm\theta)=\frac{Z(\bm\theta_B)}{Z(\bm\theta_A)}Z(\bm\theta_A)
\end{equation}
一个简单的办法就是使用Monte Carlo方法估计partition function.
\begin{equation}\begin{split}
Z_1&=\int\tilde p(\bm x)d\bm x\\
&=\int\frac{p_0(\bm x)}{p_0(\bm x)}\tilde p_1(\bm x)d\bm x\\
&=Z_0\int p_0(\bm x)\frac{\tilde p_1(\bm x)}{\tilde p_0(\bm x)}d\bm x\\
\hat Z_1&=\frac{Z_0}{K}\sum_{k=1}^K\frac{\tilde p_1(\bm x^{(k)})}{\tilde p_0(\bm x^{(k)})}\quad\text{s.t.}\,\bm x^{(k)}\sim p_0
\end{split}\end{equation}
如果$p_0$和$p_1$的分布情况比较接近,那么上式是一种比较有效的partition function的估计方式.但是在大部分情况下,$p_1$分布很复杂并且维度很高.那么这就需要借助于一定的方法去弥补$p_0$和$p_1$之间的差异.

\subsection{Annealed Importance Sampling}

在$DL(p_0\|p_1)$很大时,考虑使用\textit{annealed importance sampling}(AIS).考虑分布序列$p_{\eta_0},p_{\eta_1},\cdots,p_{\eta_n}$,其中,$0=\eta_0<\eta_1<\cdots<\eta_{n-1}<\eta_n=1$.

中间分布需要针对具体问题进行特别设计,通常的做法是使用权重几何平均
\begin{equation}
p_{\eta_j}\propto p_1^{n_j}p_0^{1-n_j}
\end{equation}

\subsection{Bridge Sampling}

\textit{Bridge sampling}的原理是,借助于单个分布$p_\ast$,去对一个partition function的分布和一个未知partition function的分布做桥接.
\begin{equation}\label{eq:bridge_sampling}
\frac{Z_1}{Z_0}\approx\sum_{k=1}^K\frac{\tilde p_\ast(\bm x_0^{(k)})}{\tilde p_0(\bm x_0^{(k)})}\Big/\frac{\tilde p_\ast(\bm x_0^{(k)})}{\tilde p_1(\bm x_0^{(k)})}
\end{equation}
$p_\ast$选择策略是尽可能覆盖$p_0$和$p_1$.式\ref{eq:bridge_sampling}通常采用迭代方法最终获取精确值$r$.

\paragraph{Linked importance sampling}当$p_0$和$p_1$的差距过大时考虑使用AIS.

\paragraph{Estimating the partition function while training}另外一种策略是在训练过程中对partition function进行估计.
\chapter{Approximate Inference}

概率模型训练的难点在于推断.推断的目的是为了求得\textbf{目标变量的边际分布}或者\textbf{以某些可观测变量为条件的条件分布}.深度学习中推断问题比较困难是由于隐变量间存在交互关系.

\section{Inference as Optimization}

精确推断可以归结为一个优化问题,而近似推断则是精确推断的估计.

假设概率模型包含可观测变量$\mathbf v$和隐变量$\mathbf h$,我们的目的是计算可观测变量的对数概率$\log\,p(\mathbf v;\theta)$.在有些情况下边际化$\mathbf h$很困难,可以计算$\log\,p(\mathbf v;\theta)$的下限$\mathcal L(\mathbf v,\theta,q)$,即\textit{evidence lower bound}(ELBO).ELBO的另一个更常见的名称是\textit{negative variation free energy}.
\begin{equation}
\mathcal L(\mathbf v,\theta,q)=\log\,p(\mathbf v;\theta)-D_{KL}\Big(q(\mathbf{h|v})\big\|p(\mathbf{h|v;\theta})\Big)
\end{equation}

因此我们可以通过最大化$\mathcal L$逼近真实分布进行推断.也可以通过控制$\mathcal L$的松紧程度灵活控制计算量.

\section{Expectation Maximization}

\textit{Expectation maximization}(EM)算法是一种基于最大化$\mathcal L$的推断算法.它适用于训练包含隐变量的模型.但它并不是近似推断的方法,而是学习近似后验概率的方法.

EM算法可以分两步,第一步通过选择分布$q$来最大化$\mathcal L$;第二步通过选择$\theta$来最大化$\mathcal L$\footnote{具体算法步骤可见原书P$634$页下方.}.

EM算法有两点本质
\begin{enumerate}
    \item 学习过程有一个基本的结构,那就是通过更新模型参数来提升完备数据集的似然度,其中所有的缺失变量可以通过后验概率进行估计填充;
    \item 在EM算法中,我们在更新了$\theta$值之后,仍然可以沿用同一个$q$的值进行训练.
\end{enumerate}
\chapter{Deep Generative Models}

\section{Boltzmann Machines}

\textit{Boltzmann机}是一种基于能量的模型,这也就意味着我们需要使用能量函数来定义联合概率分布
\begin{equation}
P(\bm x)=\frac{\exp(-E(\bm x))}{Z}
\end{equation}
其中$Z$是partition函数,$E$是能量函数
\begin{equation}
E(\bm x)=\bm{-x^TUx-b^Tx}
\end{equation}
其中,$\bm U$是模型参数的权重矩阵,$\bm b$是偏置参数的向量.

从上式可以看出,某个单元的概率可能由其它单元的概率线性组合而成,这一个局限性.而引入隐变量可以打破这种局限性.如果引入隐变量,可以实现对离散变量的概率分布进行全估计.

引入了隐变量之后,能量函数定义为
\begin{equation}
E(\bm x)=\bm{-v^TRv-v^TWh-h^TSh-b^Tv-c^Th}
\end{equation}

\paragraph{Boltamann Machine Learning} Boltamann机学习算法通常基于最大似然度,需要对partition函数进行估计.在学习过程中,连接两个单元的权重更新仅仅与这两个单元的统计量(如$P_{model}(\bm v)$,$\hat P_{data}(\bm v)P_{model}(\bm{h|v})$)有关联.也就是说,学习时局部性的,这在生物学上也是可以接受的.

\section{Restricted Boltzmann Machines}

与Boltzmann机一样,RBM也是基于能量的模型
\begin{equation}
P(\mathbf v=\bm v,\mathbf h=\bm h)=\frac{1}{Z}\exp(-E(\bm{v,h}))
\end{equation}
其中能量$E$定义为
\begin{equation}
E(\bm{v,h})=\bm{-b^Tv-c^Th-v^TWh}
\end{equation}
没有了$\bm v$与$\bm v$,$\bm h$与$\bm h$的交互项;$Z$是partition函数
\begin{equation}
Z=\sum_{\bm v}\sum_{\bm h}\exp\{-E(\bm{v,h})\}
\end{equation}

\subsection{Conditional Distributions}

虽然$P(\bm v)$很难计算,但是RBM的二部图结构特点使得$P(\mathbf{h|v})$与$P(\mathbf{v|h})$可以相对容易地进行计算以及采样.

原书P$658$推理过程可得,
\begin{equation}
P(\bm{h|v})=\prod_{j=1}^{n_h}\sigma\Big((2\bm h-1)\odot(\bm c+\bm W^T\bm v)\Big)_j
\end{equation}
\begin{equation}
P(\bm{v|h})=\prod_{i=1}^{n_v}\sigma\Big((2\bm h-1)\odot(\bm b+\bm W^T\bm h)\Big)_i
\end{equation}

\subsection{Training Restricted Boltzmann Machines}

RBM可以使用第\ref{ch:partition_function}章中任意一种具有难以计算的partition函数的训练方式(如CD,SML,PCD,ratio matching等)进行训练.

\section{Deep Belief Networks}

\textit{深度信念网络}(deep belief network, DBN)是具有多层隐变量的生成式模型.隐变量是二值离散的,可见单元既可以是离散的,也可以是连续的.图的最上两层是无向的,其余各层是有向的,箭头均指向离数据最近的层.因此DBN的图模型是有向图和无向图的混合.特别地,仅由一层隐层的模型是RBM.

深度信念网络的采样方法是,最上两层隐层使用Gibbs采样,后续层上使用ancestral采样.

深度信念网络使用逐层训练的方法进行训练
\begin{enumerate}
    \item 使用constractive divergence或者stochastic maximum likelihood方法最大化$\mathbb E_{\mathbf v\sim p_{data}}\log p(\bm v)$来训练一个RBM
    \item 第二个RBM使用近似地最大化
    \begin{equation}
    \mathbb E_{\mathbf v\sim p_{data}}\mathbb E_{\mathbf h^{(1)}\sim p^{(1)}(\bm h^{(1)}|\bm v)}\log p^{(2)}(\bm h^(1))
    \end{equation}
    其中$p^{(1)}$是第一个RBM所代表的概率分布,$p^{(2)}$是第二个RBM所代表的概率分布.
    \item 重复以上过程,直到深度信念网络达到预期的层数.
\end{enumerate}
在大多数应用中,不对深度信念网络进行整体训练,但可以通过wake-sleep算法进行参数的精细调节.

虽然深度信念网络是生成式模型,但可以用来提升分类模型的效果.我们可以使用深度信念网络的权重定义一个多层感知器网络(MLP)
\begin{equation}\begin{split}
\bm h^{(1)}&=\sigma(b^{(1)}+\bm v^T\bm W^{(1)})\\
\bm h^{(l)}&=\sigma(b_i^{(l)}+\bm h^{(l-1)T}\bm W^{(l)})
\end{split}\end{equation}
但是这样定义的多层感知器网络忽略了许多深度信念网络中的重要交互.

深度信念网络特制在最深层使用无向连接而在其它层使用指向数据的有向连接的模型.需要特别注意与信念网络进行区分,因为信念网络有时特指有向图模型.
\end{document}